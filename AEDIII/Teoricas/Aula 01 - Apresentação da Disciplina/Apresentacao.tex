\documentclass[aspectratio=169]{beamer}
\usetheme{Boadilla}
%\usetheme{Warsaw}
%\setbeamercovered{transparent}
\beamertemplatetransparentcoveredhigh
\usepackage[portuges]{babel}
\usepackage[utf8]{inputenc}
\usepackage{lmodern}
\usepackage[T1]{fontenc}
\usepackage{hyperref} 


\newcommand{\eng}[1]{\textsl{#1}}
\newcommand{\cod}[1]{\texttt{#1}}

\title[Apresentação da Disciplina]{Algoritmo e Estrutura de Dados III}
\subtitle{Apresentação da Disciplina}
\author[Frederico Santos de Oliveira]{prof. Frederico Santos de Oliveira}
\institute[UFMT]{Universidade Federal de Mato Grosso\\ Faculdade de Engenharia}
\date{}

\begin{document}

\begin{frame}[plain]
  \titlepage
\end{frame}

%\section*{Roteiro}

\begin{frame}
  \frametitle{Agenda}
  \tableofcontents
\end{frame}

%\section{Introdução}
%
%%\begin{frame}{Quem sou Eu}
%%\begin{columns}[T] % align columns
%%\begin{column}{.30\textwidth}
%%{\bf Frederico S. Oliveira}
%%\begin{figure}[h]
%%\includegraphics[width=3.5cm]{imagens/perfil.png}
%%\end{figure}
%%\end{column}%
%%\hfill%
%%\begin{column}{.64\textwidth}
%%\begin{itemize}
%%\item Bacharel e Mestre em Ciência da Computação (UFLA). 
%%\item Doutorando em Inteligência Artificial (UFG). 
%%\item Professor UFMT:
%%\begin{itemize}
%%\item Agoritmos e Estrutura de Dados I
%%\item Agoritmos e Estrutura de Dados II
%%\item Inteligência Artificial
%%\end{itemize}
%%\item \href{http://freds0.github.io}{http://freds0.github.io}
%%\item fred.santos.oliveira@gmail.com
%%\end{itemize}
%%\end{column}%
%%
%%\end{columns}
%%\end{frame}

%%%%%%%%%%%%%%%%%%%%%%%%%%%%%%%%%%%%%%%%%%%%%%%%%%%%%%%%%%%%%%%%%%%%%%%%%%%%%%%%%%%%%%%%%%%%%%%%%%
\section{Ementa da disciplina}
%%%%%%%%%%%%%%%%%%%%%%%%%%%%%%%%%%%%%%%%%%%%%%%%%%%%%%%%%%%%%%%%%%%%%%%%%%%%%%%%%%%%%%%%%%%%%%%%%%

\begin{frame}
\frametitle{Ementa da disciplina}
Árvores. Fila de prioridades. Árvores binárias de Busca. Árvores de Altura Balanceada. Árvores B e indexação em arquivos. Algoritmos em Grafos: busca, numeração topológica, árvore geradora mínima e caminhos mínimos. Espalhamento. Processamento de Cadeias (busca de padrões e compactação de Dados).
\end{frame}

%%%%%%%%%%%%%%%%%%%%%%%%%%%%%%%%%%%%%%%%%%%%%%%%%%%%%%%%%%%%%%%%%%%%%%%%%%%%%%%%%%%%%%%%%%%%%%%%%%

\begin{frame}
\frametitle{Objetivo Geral}
\begin{itemize}
 \item Apresentar estruturas de dados avançadas como árvores e grafos e mostrar suas aplicações em problemas.
\end{itemize}
\end{frame}

%%%%%%%%%%%%%%%%%%%%%%%%%%%%%%%%%%%%%%%%%%%%%%%%%%%%%%%%%%%%%%%%%%%%%%%%%%%%%%%%%%%%%%%%%%%%%%%%%%

\begin{frame}
\frametitle{Objetivos Específicos}
\begin{itemize}
 \item Apresentar as estruturas de dados Heap, árvores binárias, árvores balanceadas (AVL, B e Rubro Negra), grafos, filas de prioridade e tabelas hash.
 \item Desenvolver os principais algoritmos para manipulação dessas estruturas de dados.
\end{itemize}  
\end{frame}

%%%%%%%%%%%%%%%%%%%%%%%%%%%%%%%%%%%%%%%%%%%%%%%%%%%%%%%%%%%%%%%%%%%%%%%%%%%%%%%%%%%%%%%%%%%%%%%%%%
\section{Conteúdo Programático}
%%%%%%%%%%%%%%%%%%%%%%%%%%%%%%%%%%%%%%%%%%%%%%%%%%%%%%%%%%%%%%%%%%%%%%%%%%%%%%%%%%%%%%%%%%%%%%%%%%

\begin{frame}
\frametitle{Conteúdo Programático}
\begin{itemize}
  \item Unidade I - Tabelas Hash
  \begin{itemize}
  \item Tabelas Hash  
  \end{itemize}
   \item Unidade II - Árvores
  \begin{itemize}
  \item Introdução à Árvores
  \item Árvores Binárias
  \item Árvores Binárias de Busca   
  \end{itemize}
\end{itemize}
 \end{frame}

%%%%%%%%%%%%%%%%%%%%%%%%%%%%%%%%%%%%%%%%%%%%%%%%%%%%%%%%%%%%%%%%%%%%%%%%%%%%%%%%%%%%%%%%%%%%%%%%%%

\begin{frame}
\frametitle{Conteúdo Programático}
\begin{itemize}
  \item Unidade III - Fila de Prioridade (Heap)
  \begin{itemize}
  \item Heap
  \end{itemize}
 
  \item Unidade IV - Árvores Balanceadas 
  \begin{itemize}
  \item Árvores AVL 
  \item Árvores Rubro Negra  
  \end{itemize}
 \end{itemize}
 \end{frame}

%%%%%%%%%%%%%%%%%%%%%%%%%%%%%%%%%%%%%%%%%%%%%%%%%%%%%%%%%%%%%%%%%%%%%%%%%%%%%%%%%%%%%%%%%%%%%%%%%%%

\begin{frame}
\frametitle{Conteúdo Programático}
\begin{itemize}
  \item Unidade V - Árvores Balanceadas (Avançadas)
  \begin{itemize}
  \item Árvores B e variações    
  \end{itemize}
  
  \item Unidade VI - Grafos
  \begin{itemize}
  \item Introdução à Grafos
  \item Representação de Grafos (Matriz e Lista de Adjacência)
  \end{itemize}  
\end{itemize}
 \end{frame}
 
%%%%%%%%%%%%%%%%%%%%%%%%%%%%%%%%%%%%%%%%%%%%%%%%%%%%%%%%%%%%%%%%%%%%%%%%%%%%%%%%%%%%%%%%%%%%%%%%%%

\begin{frame}
\frametitle{Conteúdo Programático}
\begin{itemize}
   \item Unidade VII - Percurso em Grafos
  \begin{itemize}
  \item Busca em Largura 
  \item Busca em Profundidade
  \end{itemize}
\end{itemize}
 \end{frame}

%%%%%%%%%%%%%%%%%%%%%%%%%%%%%%%%%%%%%%%%%%%%%%%%%%%%%%%%%%%%%%%%%%%%%%%%%%%%%%%%%%%%%%%%%%%%%%%%%%

\begin{frame}
\frametitle{Conteúdo Programático}
\begin{itemize}
  \item Unidade VIII - Caminho Mínimo em Grafos
  \begin{itemize}
  \item Algoritmo de Kruskal
  \item Algoritmo de Dijkstra
  \end{itemize}
   \item Unidade IX - Caminho Mínimo em Grafos
  \begin{itemize}
  \item Algoritmo de Bellman-Ford
  \item Algoritmo de Floyd-Warshall  
  \end{itemize}
  
\end{itemize}
 \end{frame}

%%%%%%%%%%%%%%%%%%%%%%%%%%%%%%%%%%%%%%%%%%%%%%%%%%%%%%%%%%%%%%%%%%%%%%%%%%%%%%%%%%%%%%%%%%%%%%%%%%
\section{Metodologia}
%%%%%%%%%%%%%%%%%%%%%%%%%%%%%%%%%%%%%%%%%%%%%%%%%%%%%%%%%%%%%%%%%%%%%%%%%%%%%%%%%%%%%%%%%%%%%%%%%%

\begin{frame}
\frametitle{Metodologia}
Metodologia
\begin{itemize}
 \item Vídeo-Aulas assíncronas disponibilizadas on-line
 \item Aulas síncronas para esclarecimento de dúvidas.
 \item Desenvolvimento de exercícios.
 \item Uso do Ambiente Virtual de Aprendizagem (www.ava.ufmt.br)
\end{itemize}
\end{frame}

%%%%%%%%%%%%%%%%%%%%%%%%%%%%%%%%%%%%%%%%%%%%%%%%%%%%%%%%%%%%%%%%%%%%%%%%%%%%%%%%%%%%%%%%%%%%%%%%%%
\section{Avaliações}
%%%%%%%%%%%%%%%%%%%%%%%%%%%%%%%%%%%%%%%%%%%%%%%%%%%%%%%%%%%%%%%%%%%%%%%%%%%%%%%%%%%%%%%%%%%%%%%%%%

\begin{frame}
\frametitle{Avaliações}
Avaliações
\begin{itemize}
 \item Lista de Exercícios ($Ex_1$, $Ex_2$ ... $Ex_n$).
 \item Trabalhos Práticos (TP1, TP2).
\end{itemize}
\end{frame}

%%%%%%%%%%%%%%%%%%%%%%%%%%%%%%%%%%%%%%%%%%%%%%%%%%%%%%%%%%%%%%%%%%%%%%%%%%%%%%%%%%%%%%%%%%%%%%%%%%

\begin{frame}
\frametitle{Avaliações}
Avaliações
\begin{itemize}
 \item As médias das listas de exercícios (LE) e trabalhos práticos (TP) serão calculadas pelas fórmulas: 
 \begin{equation}
   LE = \frac { \sum_{i=1}^{n} Ex_i } {n}   \textrm{ e }  TP = \frac{(TP1+ TP2)}{2} \nonumber.
 \end{equation}
 \item A média final do aluno será calculada combinando as médias das listas de exercícios ($LE$) e dos trabalhos práticos ($TP$), da seguinte forma: 
 \begin{equation}
   MF = (0,7TP + 0,3LE) \geq 5,0, \nonumber
 \end{equation}
  para aprovação.
\end{itemize}
\end{frame}

%%%%%%%%%%%%%%%%%%%%%%%%%%%%%%%%%%%%%%%%%%%%%%%%%%%%%%%%%%%%%%%%%%%%%%%%%%%%%%%%%%%%%%%%%%%%%%%%%%

\begin{frame}
\frametitle{Avaliações}
A média final necessária para ser aprovado é 5.0:
\begin{itemize}
 \item Se $MF \geq 5.0$ {\bf Aprovado}
 \item Caso contrário, {\bf Reprovado}
\end{itemize}
\end{frame}

%%%%%%%%%%%%%%%%%%%%%%%%%%%%%%%%%%%%%%%%%%%%%%%%%%%%%%%%%%%%%%%%%%%%%%%%%%%%%%%%%%%%%%%%%%%%%%%%%%
\section{Bibliografia}
%%%%%%%%%%%%%%%%%%%%%%%%%%%%%%%%%%%%%%%%%%%%%%%%%%%%%%%%%%%%%%%%%%%%%%%%%%%%%%%%%%%%%%%%%%%%%%%%%%

\begin{frame}
\frametitle{Bibliografia Utilizada}
\begin{itemize}
 \item ZIVIANI, N. Projeto de Algoritmos: com implementações em Pascal e C.
Cengage Learning, 2010.
 \item CORMEN, T. H. Algoritmos - Teoria e Prática. Elsevier, 2009.
 \item DEITEL, H. C - Como programar. Pearson, 2011.
 \item SEDGEWICK, R. Algorithms. 4th ed. Addison-Wesley, 2011.
\end{itemize}
\end{frame}

%%%%%%%%%%%%%%%%%%%%%%%%%%%%%%%%%%%%%%%%%%%%%%%%%%%%%%%%%%%%%%%%%%%%%%%%%%%%%%%%%%%%%%%%%%%%%%%%%%

\begin{frame}[plain]
  \titlepage
\end{frame}


\end{document}
