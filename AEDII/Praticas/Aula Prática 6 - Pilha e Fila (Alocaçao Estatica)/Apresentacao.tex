\documentclass[aspectratio=169]{beamer}
\usetheme{Boadilla}
%\usetheme{Warsaw}
%\setbeamercovered{transparent}
\beamertemplatetransparentcoveredhigh
\usepackage[portuges]{babel}
\usepackage[utf8]{inputenc}
\usepackage{lmodern}
\usepackage[T1]{fontenc}
\usepackage[portuguese, linesnumbered, vlined, titlenumbered, ruled]{algorithm2e}
\SetKwRepeat{Registro}{registro \{}{\}}%
\usepackage{hyperref} 

\usepackage{xcolor}

\definecolor{codegreen}{rgb}{0,0.6,0}
\definecolor{codegray}{rgb}{0.5,0.5,0.5}
\definecolor{codepurple}{rgb}{0.58,0,0.82}
\definecolor{backcolour}{rgb}{0.95,0.95,0.92}

\usepackage{listings}
\lstdefinestyle{CStyle}{
    language=C++,
    backgroundcolor=\color{backcolour},   
    commentstyle=\color{codegreen},
    keywordstyle=\color{magenta},
    numberstyle=\tiny\color{codegray},
    stringstyle=\color{codepurple},
    basicstyle=\ttfamily\footnotesize,
    breakatwhitespace=false,         
    breaklines=true,                 
    keepspaces=true,                 
    numbers=left,       
    numbersep=5pt,                  
    showspaces=false,                
    showstringspaces=false,
    showtabs=false,                  
    tabsize=2,
}

\newcommand{\eng}[1]{\textsl{#1}}
\newcommand{\cod}[1]{\texttt{#1}}

\title[Aula Prática Pilha, Fila e Lista]{Pilha, Fila e Lista\\
   Algoritmos e Estrutura de Dados}
\subtitle{Pilha, Fila e Lista\\(Alocação Estática)}
\author[Frederico Santos de Oliveira]{prof. Frederico Santos de Oliveira}
\institute[UFMT]{Universidade Federal de Mato Grosso\\ Instituto de Engenharia}
\date{}

\begin{document}

\begin{frame}[plain]
  \titlepage
\end{frame}

%%%%%%%%%%%%%%%%%%%%%%%%%%%%%%%%%%%%%%%%%%%%%%%%%%%%%%%%%%%%%%%%%%%%%%%%%%%%%%%%%%%%%%%%%%%%%%%%%%%%%

\begin{frame}
  \frametitle{Agenda}
  \tableofcontents
\end{frame}

%%%%%%%%%%%%%%%%%%%%%%%%%%%%%%%%%%%%%%%%%%%%%%%%%%%%%%%%%%%%%%%%%%%%%%%%%%%%%%%%%%%%%%%%%%%%%%%%%%%%%
\section{Exercício 1}
%%%%%%%%%%%%%%%%%%%%%%%%%%%%%%%%%%%%%%%%%%%%%%%%%%%%%%%%%%%%%%%%%%%%%%%%%%%%%%%%%%%%%%%%%%%%%%%%%%%%%

\begin{frame}
\frametitle{Exercício 1}
Implemente uma pilha utilizando alocação estática. As operações básicas são:
\begin{enumerate}
 \item CriaPilhaVazia($S$) 
 \item PilhaVazia($S$)
 \item Empilha($S$,$x$)
 \item Desempilha($S$) 
\end{enumerate}
\end{frame}

%%%%%%%%%%%%%%%%%%%%%%%%%%%%%%%%%%%%%%%%%%%%%%%%%%%%%%%%%%%%%%%%%%%%%%%%%%%%%%%%%%%%%%%%%%%%%%%%%%%%%

\begin{frame}[fragile]
\frametitle{Exercício 1}
Para isso, o primeiro passo é criar a estrutura de dados chamada Pilha, que possui uma variável chamada Topo, que aponta para o índice do elemento no topo da pilha, e um vetor de itens.
\begin{algorithm}[H]
\caption{Pilha} 
\label{Nodo}
\Inicio{
 \Registro{Pilha}{
    Inteiro: itens[MAX]; \\
    Inteiro Topo; 
  }
}
\end{algorithm} 
\end{frame}

%%%%%%%%%%%%%%%%%%%%%%%%%%%%%%%%%%%%%%%%%%%%%%%%%%%%%%%%%%%%%%%%%%%%%%%%%%%%%%%%%%%%%%%%%%%%%%%%%%%%%

\begin{frame}[fragile]
\frametitle{Exercício 1}
Na linguagem C, a implementação fica conforme o código a seguir:
\begin{lstlisting}[style=CStyle]
#define MAX 1000

typedef struct {
  int itens[MAX];
  int topo;
} pilha;
\end{lstlisting}  
Assim, definindo uma constante chamada MAX com valor igual a 1000, será possível incluir no máximo 1000 itens na nossa pilha.
\end{frame}

%%%%%%%%%%%%%%%%%%%%%%%%%%%%%%%%%%%%%%%%%%%%%%%%%%%%%%%%%%%%%%%%%%%%%%%%%%%%%%%%%%%%%%%%%%%%%%%%%%%%%

\begin{frame}[fragile]	{Pilhas}{CriaPilhaVazia}
A seguir, tem-se o pseudo-código referente à função CriaPilhaVazia:
\begin{algorithm}[H]
\caption{CriaPilhaVazia} 
\label{CriaPilhaVazia}
\Entrada{Pilha $S$}
\Inicio{
  $S$.topo $\leftarrow$ 0 \\
}
\end{algorithm}
\end{frame}

%%%%%%%%%%%%%%%%%%%%%%%%%%%%%%%%%%%%%%%%%%%%%%%%%%%%%%%%%%%%%%%%%%%%%%%%%%%%%%%%%%%%%%%%%%%%%%%%%%%%%

\begin{frame}[fragile]{Pilhas}{CriaPilhaVazia}

Sua implementação em C, fica da seguinte forma:
\begin{lstlisting}[style=CStyle]
void CriaPilhaVazia(pilha *S) { 
  S -> topo = 0; 
} 
\end{lstlisting}  
Observe que a função recebe como parâmetro um ponteiro do tipo \verb|pilha|. Ao chamar essa função, caso tenha declarado uma variável estática do tipo pilha, será necessário utilizar o operador de endereçamento (\&):

\begin{lstlisting}[style=CStyle]
pilha S;
CriaPilhaVazia(&S);
\end{lstlisting}  
Caso utilize um ponteiro, não é necessário utilizar o operador de endereçamento:
\begin{lstlisting}[style=CStyle]
pilha *S  = malloc(sizeof(pilha));
CriaPilhaVazia(S);
\end{lstlisting}  
\end{frame}


%%%%%%%%%%%%%%%%%%%%%%%%%%%%%%%%%%%%%%%%%%%%%%%%%%%%%%%%%%%%%%%%%%%%%%%%%%%%%%%%%%%%%%%%%%%%%%%%%%%%%

\begin{frame}[fragile]{Pilhas}{PilhaVazia}
Agora, vamos implementar a função que verifica se uma pilha está vazia.

\scalebox{0.8}{
\begin{algorithm}[H]
\caption{PilhaVazia} 
\label{PilhaVazia}
\Entrada{Pilha $S$}
\Saida{Booleano informando se $S$ está vazia}
\Inicio{
  \Se { ($S$.topo = 0) } {
    \Retorna Verdadeiro \\
    }
  \Senao {
    \Retorna Falso
  } 
}
\end{algorithm}
}

A implementação na linguagem C fica da seguinte forma:

\begin{lstlisting}[style=CStyle]
int PilhaVazia(pilha S) { 
  return (S.topo == 0); 
} 
\end{lstlisting}  
\end{frame}

%%%%%%%%%%%%%%%%%%%%%%%%%%%%%%%%%%%%%%%%%%%%%%%%%%%%%%%%%%%%%%%%%%%%%%%%%%%%%%%%%%%%%%%%%%%%%%%%%%%%%

\begin{frame}[fragile]{Pilhas}{Empilhar}
A próxima função é \verb|Empilha|:
% \scalebox{0.5}{
\begin{algorithm}[H]
\caption{Empilhar} 
\label{Empilhar}
\Entrada{Pilha $S$, item $x$ a ser empilhado}
\Inicio{
  \Se {($S$.topo = $S$.tamanhoMáximo)} {
      {\bf Erro } {\it overflow:} pilha cheia.
  }
  \Senao {
      $S$.itens[$S$.topo] $\leftarrow$ x \\
      $S$.topo $\leftarrow$ $S$.topo + 1 \\      
  }
}
\end{algorithm}
% }  
\end{frame}

%%%%%%%%%%%%%%%%%%%%%%%%%%%%%%%%%%%%%%%%%%%%%%%%%%%%%%%%%%%%%%%%%%%%%%%%%%%%%%%%%%%%%%%%%%%%%%%%%%%%%

\begin{frame}[fragile]{Pilhas}{Empilhar}
A implementação na linguagem C fica da seguinte forma:

\begin{lstlisting}[style=CStyle]
void Empilha(int x, pilha *S) { 
  if (S -> topo == MAX) 
    printf(" Erro   pilha esta  cheia\n");
  else  { 
    S->itens[S->topo] = x;   
    S->topo++; 
  }
} 
\end{lstlisting}  
Observe que essa função recebe um ponteiro como parâmetro de entrada. 
\end{frame}

%%%%%%%%%%%%%%%%%%%%%%%%%%%%%%%%%%%%%%%%%%%%%%%%%%%%%%%%%%%%%%%%%%%%%%%%%%%%%%%%%%%%%%%%%%%%%%%%%%%%%

\begin{frame}[fragile]{Pilhas}{Empilhar}
Caso tenha declarado a pilha S como uma variável estática, ao invocar a função \verb|Empilha|, a solução é utilizar o {\it operador de endereço} (\&):

\begin{lstlisting}[style=CStyle]
pilha S;
Empilha(10, &S) 

\end{lstlisting}  
Dessa forma, é passado como parâmetro o endereço da variável $S$, que será armazenado em um ponteiro.
\end{frame}


%%%%%%%%%%%%%%%%%%%%%%%%%%%%%%%%%%%%%%%%%%%%%%%%%%%%%%%%%%%%%%%%%%%%%%%%%%%%%%%%%%%%%%%%%%%%%%%%%%%%%

\begin{frame}[fragile]{Pilhas}{Desempilhar}
A próxima função é \verb|Desempilha|:
% \scalebox{0.5}{
\begin{algorithm}[H]
\caption{Desempilhar} 
\label{Desempilhar}
\Entrada{Pilha $S$}
\Saida{Elemento desempilhado}
\Inicio{
  \Se { (PilhaVazia($S$)) } {
      \Retorna {\it underflow: } pilha vazia.
  } 
  \Senao {
      $x \leftarrow S$.itens[ $S$.topo - 1]\\
      $S$.topo $\leftarrow S$.topo - 1\\
      \Retorna $x$      
  }
}
\end{algorithm}
% }  
\end{frame}

%%%%%%%%%%%%%%%%%%%%%%%%%%%%%%%%%%%%%%%%%%%%%%%%%%%%%%%%%%%%%%%%%%%%%%%%%%%%%%%%%%%%%%%%%%%%%%%%%%%%%

\begin{frame}[fragile]{Pilhas}{Desempilhar}
A implementação na linguagem C fica da seguinte forma:

\begin{lstlisting}[style=CStyle]
void Desempilha(pilha *S, int *item) { 
  if (PilhaVazia(*S)) 
    printf(" Erro   pilha esta  vazia\n");
  else 
  { 
    *item = S->itens[S->topo - 1]; 
    S->topo--;     
  }
} 
\end{lstlisting}  
Observe que, assim como a função \verb|Empilha|, essa função também recebe um ponteiro como parâmetro de entrada, e, além disso, também recebe um ponteiro no qual ficará armazenado o elemento desempilhado. 
\end{frame}

%%%%%%%%%%%%%%%%%%%%%%%%%%%%%%%%%%%%%%%%%%%%%%%%%%%%%%%%%%%%%%%%%%%%%%%%%%%%%%%%%%%%%%%%%%%%%%%%%%%%%

\begin{frame}[fragile]{Pilhas}{Desempilhar}
Assim, para utilizar a função desempilha, pode-se utilizar o operador de endereçamento, conforme o código a seguir:

\begin{lstlisting}[style=CStyle]
int main() { 
    // Declaracao de variaveis
	pilha S; 
	int item = 0;
	// Cria uma pilha vazia
	CriaPilhaVazia(&S);
    // Empilha elemento 10 na pilha
	Empilha(10, &S);
	// Desempilha elemento
	Desempilha(&S, &item);
	printf("Desempilhou %d.", item)	;
	
	return(0);
} 
\end{lstlisting}  
\end{frame}

%%%%%%%%%%%%%%%%%%%%%%%%%%%%%%%%%%%%%%%%%%%%%%%%%%%%%%%%%%%%%%%%%%%%%%%%%%%%%%%%%%%%%%%%%%%%%%%%%%%%%

\begin{frame}[fragile]{Pilhas}{Desempilhar}
Também é possível utilizar um ponteiro do tipo \verb|pilha|, e assim, não é necessário utilizar o operador de endereçamento:

\begin{lstlisting}[style=CStyle]
int main() { 
    // Declaracao de variaveis
	pilha *S = malloc(sizeof(pilha)); 
	int item = 0;
	// Cria uma pilha vazia
	CriaPilhaVazia(S);
    // Empilha elemento 10 na pilha
	Empilha(10, S);
	// Desempilha elemento
	Desempilha(S, &item);
	printf("Desempilhou %d.", item)	;
	// Desaloca memoria
	free(S);
	
	return(0);
} 
\end{lstlisting}  
\end{frame}


\section{Exercício 2}

\begin{frame}
\frametitle{Exercício 2}
Implemente uma fila utilizando alocação estática. As operações básicas são:
\begin{enumerate}
 \item CriaFilaVazia($Q$) 
 \item FilaVazia($Q$)
 \item Enfileira($Q$,$x$)
 \item Desenfileira($Q$) 
\end{enumerate}
\end{frame}

%%%%%%%%%%%%%%%%%%%%%%%%%%%%%%%%%%%%%%%%%%%%%%%%%%%%%%%%%%%%%%%%%%%%%%%%%%%%%%%%%%%%%%%%%%%%%%%%%%%%%

\begin{frame}[fragile]
\frametitle{Exercício 2}
Assim como no exercício anterior, o primeiro passo é criar a estrutura de dados chamada Fila, que possui uma variável chamada \verb|início| e outra chamada \verb|fim|, e um vetor de \verb|itens|.
\begin{algorithm}[H]
\caption{Fila} 
\label{Nodo}
\Inicio{
 \Registro{Fila}{
    Inteiro: itens[MAX]; \\
    Inteiro início, fim; 
  }
}
\end{algorithm} 
\end{frame}

%%%%%%%%%%%%%%%%%%%%%%%%%%%%%%%%%%%%%%%%%%%%%%%%%%%%%%%%%%%%%%%%%%%%%%%%%%%%%%%%%%%%%%%%%%%%%%%%%%%%%

\begin{frame}[fragile]{Filas}
A implementação na linguagem C fica da seguinte forma:

\begin{lstlisting}[style=CStyle]
#define MAX  1000

typedef struct {
  int itens[MAX];
  int inicio, fim;
} fila;
\end{lstlisting}  
Definindo uma constante chamada MAX com valor igual a 1000, será possível incluir no máximo 1000 itens na nossa fila.
\end{frame}

%%%%%%%%%%%%%%%%%%%%%%%%%%%%%%%%%%%%%%%%%%%%%%%%%%%%%%%%%%%%%%%%%%%%%%%%%%%%%%%%%%%%%%%%%%%%%%%%%%%%%

\begin{frame}[fragile]{Fila}{CriaFilaVazia}
A próxima função é \verb|CriaFilaVazia|:
% \scalebox{0.5}{
\begin{algorithm}[H]
\caption{CriaFilaVazia} 
\label{CriaFilaVazia}
\Entrada{Fila $Q$}
\Inicio{
  $Q$.início $\leftarrow$ 0 \\
  $Q$.fim $\leftarrow$ $Q$.início
}
\end{algorithm} 
A implementação na linguagem C fica da seguinte forma:

\begin{lstlisting}[style=CStyle]
void CriaFila(fila *Q) { 
  Q->inicio = 0;
  Q->fim = Q->inicio;
} 
\end{lstlisting}  

\end{frame}


%%%%%%%%%%%%%%%%%%%%%%%%%%%%%%%%%%%%%%%%%%%%%%%%%%%%%%%%%%%%%%%%%%%%%%%%%%%%%%%%%%%%%%%%%%%%%%%%%%%%%

\begin{frame}[fragile]{Fila}{FilaVazia}
A próxima função é \verb|FilaVazia|:
% \scalebox{0.5}{
\begin{algorithm}[H]
\caption{FilaVazia} 
\label{FilaVazia}
\Entrada{Fila $Q$.}
\Inicio{
  \Se {($Q$.início = $Q$.fim)} {
      \Retorna Verdadeiro 
  }
  \Senao {
      \Retorna Falso
  }
}
\end{algorithm}
A implementação na linguagem C fica da seguinte forma:

\begin{lstlisting}[style=CStyle]
int FilaVazia(fila Q)
{ 
  return (Q.inicio == Q.fim); 
} 
\end{lstlisting}  

\end{frame}

%%%%%%%%%%%%%%%%%%%%%%%%%%%%%%%%%%%%%%%%%%%%%%%%%%%%%%%%%%%%%%%%%%%%%%%%%%%%%%%%%%%%%%%%%%%%%%%%%%%%%

\begin{frame}[fragile]{Fila}{Enfileira}
A próxima função é \verb|Enfileira|:
% \scalebox{0.5}{
\begin{algorithm}[H]
\caption{Enfileira} 
\label{Enfileira}
\Entrada{Fila $Q$, item $x$}
\Inicio{
  \Se { ( ($Q$.fim) MOD ($Q$.tamanhoMáximo) + 1 = $Q$.início)} {
      {\bf Erro }{\it overflow}: fila cheia.
  }
  \Senao {
      $Q$.itens[$Q$.fim] $\leftarrow$ x \\
      $Q$.fim $\leftarrow$ ($Q$.fim) MOD ($Q$.tamanhoMáximo) + 1 \\
  }
}
\end{algorithm}

\end{frame}

%%%%%%%%%%%%%%%%%%%%%%%%%%%%%%%%%%%%%%%%%%%%%%%%%%%%%%%%%%%%%%%%%%%%%%%%%%%%%%%%%%%%%%%%%%%%%%%%%%%%%

\begin{frame}[fragile]{Fila}{Enfileira}
A implementação na linguagem C fica da seguinte forma:

\begin{lstlisting}[style=CStyle]
void Enfileira(int x, fila *Q) { 
  if ((Q->fim+1 )% MAX  == Q->inicio)
    printf(" Erro   fila esta  cheia\n");
  else { 
    Q->itens[Q->fim - 1] = x;
    Q->fim = Q->fim % MAX + 1;
  }
} 
\end{lstlisting}  
\end{frame}

%%%%%%%%%%%%%%%%%%%%%%%%%%%%%%%%%%%%%%%%%%%%%%%%%%%%%%%%%%%%%%%%%%%%%%%%%%%%%%%%%%%%%%%%%%%%%%%%%%%%%

\begin{frame}[fragile]{Fila}{Desenfileira}
A próxima função é \verb|Desenfileira|:
% \scalebox{0.5}{
\begin{algorithm}[H]
\caption{Desenfileira} 
\label{Desenfileira}
\Entrada{Fila $Q$}
\Saida{Elemento desenfileirado}
\Inicio{
  \Se {FilaVazia($Q$)} {
      \Retorna Erro {\it underflow}: fila vazia. 
  }
  \Senao {
      $x$ $\leftarrow$ $Q$.itens[$Q$.início] \\
      $Q$.início $\leftarrow$ ($Q$.início) MOD ($Q$.tamanhoMáximo) + 1 \\
      \Retorna $x$
      
  }
}
\end{algorithm}
% }  
\end{frame}

%%%%%%%%%%%%%%%%%%%%%%%%%%%%%%%%%%%%%%%%%%%%%%%%%%%%%%%%%%%%%%%%%%%%%%%%%%%%%%%%%%%%%%%%%%%%%%%%%%%%%

\begin{frame}[fragile]{Fila}{Desenfileira}
A implementação na linguagem C fica da seguinte forma:

\begin{lstlisting}[style=CStyle]
void Desenfileira(fila *Q, int *item) { 
  if (FilaVazia(*Q))
    printf("Erro fila esta vazia\n");
  else { 
    *item = Q->itens[Q->inicio - 1];
    Q->inicio = Q->inicio % MAX + 1;
  }
} 
\end{lstlisting}  
\end{frame}

%%%%%%%%%%%%%%%%%%%%%%%%%%%%%%%%%%%%%%%%%%%%%%%%%%%%%%%%%%%%%%%%%%%%%%%%%%%%%%%%%%%%%%%%%%%%%%%%%%%%%

\begin{frame}[fragile]{Pilhas}{Desempilhar}
Observe que ao utilizar uma variável estática é necessário usar o operador de endereçamento, conforme o código a seguir:

\begin{lstlisting}[style=CStyle]
int main() { 
    // Declaracao de variaveis
	fila Q; 
	int item = 0;
	// Cria uma fila vazia
	CriaFilaVazia(&Q);
    // Enfileira elemento 10 na fila
	Enfileira(10, &Q);
	// Desenfileira elemento
	Desenfileira(&Q, &item);
	printf("Desenfileirou %d.", item)	;
	
	return(0);
} 
\end{lstlisting}  
\end{frame}

%%%%%%%%%%%%%%%%%%%%%%%%%%%%%%%%%%%%%%%%%%%%%%%%%%%%%%%%%%%%%%%%%%%%%%%%%%%%%%%%%%%%%%%%%%%%%%%%%%%%%

\begin{frame}[fragile]{Pilhas}{Desempilhar}
Também é possível utilizar um ponteiro do tipo \verb|fila|, e assim, não é necessário utilizar o operador de endereçamento:

\begin{lstlisting}[style=CStyle]
int main() { 
    // Declaracao de variaveis
	fila *Q = malloc(sizeof(fila));  
	int item = 0;
	// Cria uma fila vazia
	CriaFilaVazia(Q);
    // Enfileira elemento 10 na fila
	Enfileira(10, Q);
	// Desenfileira elemento
	Desenfileira(Q, &item);
	printf("Desenfileirou %d.", item)	;
	// Desaloca memoria
	free(Q);
	return(0);
} 
\end{lstlisting}  
\end{frame}

\section{Exercício 3}

\begin{frame}
\frametitle{Exercício 3}
Implemente uma lista utilizando alocação estática. As operações básicas são:
\begin{enumerate}
 \item CriaListaVazia($L$) 
 \item ListaVazia($L$)
 \item Inserir($L$, $x$, $p$)
 \item Remover($L$, $p$) 
\end{enumerate}
\end{frame}

%%%%%%%%%%%%%%%%%%%%%%%%%%%%%%%%%%%%%%%%%%%%%%%%%%%%%%%%%%%%%%%%%%%%%%%%%%%%%%%%%%%%%%%%%%%%%%%%%%%%%

\begin{frame}[fragile]
\frametitle{Exercício 3}
Para isso, o primeiro passo é criar a estrutura de dados chamada Lista, que possui uma variável chamada primeiro, outra chamada último, e um vetor de itens.
\begin{algorithm}[H]
\caption{Lista} 
\label{Nodo}
\Inicio{
 \Registro{Lista}{
    Inteiro: itens[MAX]; \\
    Inteiro primeiro, último 
  }
}
\end{algorithm} 
\end{frame}

%%%%%%%%%%%%%%%%%%%%%%%%%%%%%%%%%%%%%%%%%%%%%%%%%%%%%%%%%%%%%%%%%%%%%%%%%%%%%%%%%%%%%%%%%%%%%%%%%%%%%

\begin{frame}[fragile]
\frametitle{Exercício 1}
Na linguagem C, a implementação fica conforme o código a seguir:
\begin{lstlisting}[style=CStyle]
#define MAX 1000

typedef struct {
  int itens[MAX];
  int primeiro, ultimo;
} lista;
\end{lstlisting}  
Assim, definindo uma constante chamada MAX com valor igual a 1000, será possível incluir no máximo 1000 itens na nossa lista.
\end{frame}


%%%%%%%%%%%%%%%%%%%%%%%%%%%%%%%%%%%%%%%%%%%%%%%%%%%%%%%%%%%%%%%%%%%%%%%%%%%%%%%%%%%%%%%%%%%%%%%%%%%%%

\begin{frame}[fragile]{Lista}
A primeira função é \verb|CriaListaVazia|:
% \scalebox{0.5}{
\begin{algorithm}[H]
\caption{CriaListaVazia} 
\label{CriaListaVazia}
\Entrada{Lista $L$}
\Inicio{
  $L$.primeiro $\leftarrow$ 0 \\
  $L$.ultimo $\leftarrow$ $L$.primeiro \\
}
\end{algorithm}
% }  
\end{frame}

%%%%%%%%%%%%%%%%%%%%%%%%%%%%%%%%%%%%%%%%%%%%%%%%%%%%%%%%%%%%%%%%%%%%%%%%%%%%%%%%%%%%%%%%%%%%%%%%%%%%%

\begin{frame}[fragile]{Lista}{CriaListaVazia}
A implementação na linguagem C fica da seguinte forma:

\begin{lstlisting}[style=CStyle]
void CriaListaVazia(lista *l) { 
  l->primeiro = 0;
  l->ultimo = l->primeiro;
}
\end{lstlisting}  
\end{frame}

%%%%%%%%%%%%%%%%%%%%%%%%%%%%%%%%%%%%%%%%%%%%%%%%%%%%%%%%%%%%%%%%%%%%%%%%%%%%%%%%%%%%%%%%%%%%%%%%%%%%%

\begin{frame}[fragile]{Lista}{ListaVazia}
A próxima função é \verb|ListaVazia|:
\begin{algorithm}[H]
\caption{ListaVazia} 
\label{ListaVazia}
\Entrada{Lista $L$.}
\Inicio{
  \Se {($L$.primeiro = $L$.ultimo)} {
      \Retorna Verdadeiro 
  }
  \Senao {
      \Retorna Falso
  }
}
\end{algorithm}
% }  
\end{frame}

%%%%%%%%%%%%%%%%%%%%%%%%%%%%%%%%%%%%%%%%%%%%%%%%%%%%%%%%%%%%%%%%%%%%%%%%%%%%%%%%%%%%%%%%%%%%%%%%%%%%%

\begin{frame}[fragile]{Lista}{ListaVazia}
A implementação na linguagem C fica da seguinte forma:

\begin{lstlisting}[style=CStyle]
int ListaVazia(lista l) { 
  return (l.primeiro == l.ultimo);
} 
\end{lstlisting}  
\end{frame}

%%%%%%%%%%%%%%%%%%%%%%%%%%%%%%%%%%%%%%%%%%%%%%%%%%%%%%%%%%%%%%%%%%%%%%%%%%%%%%%%%%%%%%%%%%%%%%%%%%%%%

\begin{frame}[fragile]{Lista}{Insere}
A próxima função é \verb|Insere|:
\begin{algorithm}[H]
\caption{Inserir} 
\label{Inserir}
\Entrada{Lista $L$, elemento $x$, posição $p$}
\Inicio{
  \Se { ( ($L$.ultimo + 1) MOD $L$.tamanhoMáximo = $L$.primeiro )} {
      {\bf Erro }{\it overflow}: lista cheia.
  }
  \Senao {
      \Para {( $i \leftarrow L.ultimo $ até $p$) } { 
      	$L$.itens[ i+1 ] $\leftarrow$ $L$.itens[ i ] \\
      }
     	$L$.itens[ $p$ ] $\leftarrow$ x \\
     	$L$.ultimo = $L$.ultimo + 1 \\
  }
}
\end{algorithm}
% }  
\end{frame}

%%%%%%%%%%%%%%%%%%%%%%%%%%%%%%%%%%%%%%%%%%%%%%%%%%%%%%%%%%%%%%%%%%%%%%%%%%%%%%%%%%%%%%%%%%%%%%%%%%%%%

\begin{frame}[fragile]{Lista}{Insere}
A implementação na linguagem C fica da seguinte forma:

\begin{lstlisting}[style=CStyle]
void Insere(lista *l, int x, int p) { 
  if ((l->ultimo+1 )% MAX  == l->primeiro)
    printf("Lista esta cheia\n");
  else  { 
    for (int i = l->ultimo; i < p; i++) 
    	l->itens[i+1] = l->itens[i];
    l->itens[p] = x;
    l->ultimo++;
  }
} 
\end{lstlisting}  
\end{frame}

%%%%%%%%%%%%%%%%%%%%%%%%%%%%%%%%%%%%%%%%%%%%%%%%%%%%%%%%%%%%%%%%%%%%%%%%%%%%%%%%%%%%%%%%%%%%%%%%%%%%%

\begin{frame}[fragile]{Lista}{Remove}
A próxima função é \verb|Remove|:
\begin{algorithm}[H]
\caption{Remove} 
\label{Remove}
\Entrada{Lista $L$, posição $p$}
\Saida{Elemento removido}
\Inicio{
  \Se {ListaVazia($L$)} {
      \Retorna Erro {\it underflow}: lista vazia. 
  }
  \Senao {
    	$x$ $\leftarrow$ $L$.itens[ p ] \\
  	\Para {( $i \leftarrow p$ até $L.ultimo$ ) }  {
      	$L$.itens[ i ] $\leftarrow$ $L$.itens[ i+1 ] \\
     }
     $L$.ultimo = $L$.ultimo - 1 \\
     	\Retorna $x$
  }
}
\end{algorithm}
% }  
\end{frame}

%%%%%%%%%%%%%%%%%%%%%%%%%%%%%%%%%%%%%%%%%%%%%%%%%%%%%%%%%%%%%%%%%%%%%%%%%%%%%%%%%%%%%%%%%%%%%%%%%%%%%

\begin{frame}[fragile]{Lista}{Remove}
A implementação na linguagem C fica da seguinte forma:

\begin{lstlisting}[style=CStyle]
void Remove(int p, lista *l, int *item) { 
  if (ListaVazia(*l)) { 
    printf(" Erro   Lista vazia\n");
  }
  *item = l->itens[p];
  for (int aux = p; aux < l->ultimo; aux++)
    l->itens[aux] = l->itens[aux+1];
  l->ultimo--;    
} 
\end{lstlisting}  
\end{frame}

%%%%%%%%%%%%%%%%%%%%%%%%%%%%%%%%%%%%%%%%%%%%%%%%%%%%%%%%%%%%%%%%%%%%%%%%%%%%%%%%%%%%%%%%%%%%%%%%%%%%%

\begin{frame}[plain]
  \titlepage
\end{frame}


\end{document}
