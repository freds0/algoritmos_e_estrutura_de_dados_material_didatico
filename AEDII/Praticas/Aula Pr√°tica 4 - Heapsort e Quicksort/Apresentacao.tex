\documentclass[aspectratio=169]{beamer}
\usetheme{Boadilla}
%\usetheme{Warsaw}
%\setbeamercovered{transparent}
\beamertemplatetransparentcoveredhigh
\usepackage[portuges]{babel}
\usepackage[utf8]{inputenc}
\usepackage{lmodern}
\usepackage[T1]{fontenc}
\usepackage[portuguese, linesnumbered, vlined, titlenumbered, ruled]{algorithm2e}
\usepackage{hyperref} 

\usepackage{xcolor}

\definecolor{codegreen}{rgb}{0,0.6,0}
\definecolor{codegray}{rgb}{0.5,0.5,0.5}
\definecolor{codepurple}{rgb}{0.58,0,0.82}
\definecolor{backcolour}{rgb}{0.95,0.95,0.92}

\usepackage{listings}
\lstdefinestyle{CStyle}{
    language=C++,
    backgroundcolor=\color{backcolour},   
    commentstyle=\color{codegreen},
    keywordstyle=\color{magenta},
    numberstyle=\tiny\color{codegray},
    stringstyle=\color{codepurple},
    basicstyle=\ttfamily\footnotesize,
    breakatwhitespace=false,         
    breaklines=true,                 
    keepspaces=true,                 
    numbers=left,       
    numbersep=5pt,                  
    showspaces=false,                
    showstringspaces=false,
    showtabs=false,                  
    tabsize=2,
}


\newcommand{\eng}[1]{\textsl{#1}}
\newcommand{\cod}[1]{\texttt{#1}}

\title[Algoritmos de Ordenação]{Algoritmos e Estrutura de Dados II}
\subtitle{Exercício Prático\\Heapsort e Quicksort}
\author[Frederico Santos de Oliveira]{prof. Frederico Santos de Oliveira}
\institute[UFMT]{Universidade Federal de Mato Grosso\\ Instituto de Engenharia}
\date{}


\AtBeginSection[]
{
	\begin{frame}
	\frametitle{Roteiro da Aula}
	\tableofcontents[currentsection]
\end{frame}
}


\begin{document}

%%%%%%%%%%%%%%%%%%%%%%%%%%%%%%%%%%%%%%%%%%%%%%%%%%%%%%%%%%%%%%%%%%%%%%%%%%%%%%%%%%%%%%%%%%%%%%%%%%%%%%%%%%%%%

\begin{frame}[plain]
  \titlepage
\end{frame}

%%%%%%%%%%%%%%%%%%%%%%%%%%%%%%%%%%%%%%%%%%%%%%%%%%%%%%%%%%%%%%%%%%%%%%%%%%%%%%%%%%%%%%%%%%%%%%%%%%%%%%%%%%%%%
%\section*{Roteiro}
%%%%%%%%%%%%%%%%%%%%%%%%%%%%%%%%%%%%%%%%%%%%%%%%%%%%%%%%%%%%%%%%%%%%%%%%%%%%%%%%%%%%%%%%%%%%%%%%%%%%%%%%%%%%%

\begin{frame}
  \frametitle{Agenda}
  \tableofcontents
\end{frame}

%%%%%%%%%%%%%%%%%%%%%%%%%%%%%%%%%%%%%%%%%%%%%%%%%%%%%%%%%%%%%%%%%%%%%%%%%%%%%%%%%%%%%%%%%%%%%%%%%%%%%%%%%%%%%
\section{Exercício 1}
%%%%%%%%%%%%%%%%%%%%%%%%%%%%%%%%%%%%%%%%%%%%%%%%%%%%%%%%%%%%%%%%%%%%%%%%%%%%%%%%%%%%%%%%%%%%%%%%%%%%%%%%%%%%%

\begin{frame}
\frametitle{Exercício 1}
Utilizando o algoritmo de ordenação {\bf Heapsort}, ordene um vetor composto por 1.000.000, 2.000.000 e 3.000.000 valores. Para cada um, execute os seguintes testes:
\begin{itemize}
 \item vetor composto por números aleatórios;
 \item vetor composto por números em ordem crescente;
 \item vetor composto por números em ordem decrescente.
\end{itemize} 
\end{frame}

%%%%%%%%%%%%%%%%%%%%%%%%%%%%%%%%%%%%%%%%%%%%%%%%%%%%%%%%%%%%%%%%%%%%%%%%%%%%%%%%%%%%%%%%%%%%%%%%%%%%%%%%%%%%%

\begin{frame}[fragile]{Exercício 1}{Pseudocódigo}
\scalebox{0.8}{
\begin{algorithm}[H]
\caption{ConstroiHeap} 
\label{ConstroiHeap}
\Entrada{Vetor $V[i..n-1]$, raiz no nó $i$, tamanho do vetor $n$}
\Saida{Heap no vetor $V[i..n-1]$}
\Inicio{
    $maior \leftarrow i$  \CommentSty{// Inicializa maior como a raiz} \\
    $l \leftarrow 2i + 1$ \CommentSty{// Filho da esquerda} \\
    $r \leftarrow 2i + 2$ \CommentSty{// Filho da direita} \\
    \CommentSty{// Se o filho da esquerda é maior que a raiz}\\
    \Se{ ( $l < n$ AND $V[l] > V[maior]$ )} {
        $maior \leftarrow l$\\
    }
    \CommentSty{// Se filho da direita é maior que a raiz}\\
    \Se{ ($r < n$ AND $V[r] > V[maior])$ }{
        $maior \leftarrow r$\\
    }
    \CommentSty{ // Se maior não é a raiz}\\
    \Se { ($maior \neq i$)}
    {
        Troca $V[i] \leftrightarrow V[maior]$ \CommentSty{ // O maior passa a ser a raiz}\\
        ConstroiHeap(V, maior, n) \CommentSty{// Cria o heap na sub-árvore}\\
    }
}
\end{algorithm}
}
\end{frame}

%%%%%%%%%%%%%%%%%%%%%%%%%%%%%%%%%%%%%%%%%%%%%%%%%%%%%%%%%%%%%%%%%%%%%%%%%%%%%%%%%%%%%%%%%%%%%%%%%%%%%%%%%%%%%

\begin{frame}[fragile]{Exercício 1}{Solução}
\begin{lstlisting}[style=CStyle]
void constroi_heap(int *vetor, int i, int n) {
    int maior, l, r, aux;
    maior = i;
    l = 2*i + 1;
    r = 2*i + 2;
    if (l<n && vetor[l]>vetor[maior])
        maior = l;
    if ((r<n) && vetor[r]>vetor[maior]) 
        maior = r;
    if (maior!=i) {
        aux = vetor[i];
        vetor[i] = vetor[maior];
        vetor[maior] = aux;
        constroi_heap(vetor, maior, n);
    }
}
\end{lstlisting}  
\end{frame}

%%%%%%%%%%%%%%%%%%%%%%%%%%%%%%%%%%%%%%%%%%%%%%%%%%%%%%%%%%%%%%%%%%%%%%%%%%%%%%%%%%%%%%%%%%%%%%%%%%%%%%%%%%%%%

\begin{frame}[fragile]{Exercício 1}{Pseudocódigo}
\scalebox{0.95}{
\begin{algorithm}[H]
\caption{Heapsort} 
\label{Heap}
\Entrada{Vetor $V[0..n-1]$, tamanho do vetor $n$}
\Saida{Vetor $V$ ordenado}
\Inicio{
    \CommentSty{// Contrói o heap rearranjando o vetor}\\
    \Para { ( $i \leftarrow \frac{n}{2}-1$ decrescendo até $i = 0$ )} {
        ConstroiHeap(V, i, n) \\
    }
    \CommentSty{// Extrai cada elemento, um por um, do heap}\\
    \Para {( $i\leftarrow n-1$ decrescendo até $i=0$) }
    {
        \CommentSty{// Move a raiz atual para o fim do vetor.}\\
        Troca $V[0] \leftrightarrow V[i]$\\
        \CommentSty{// Chama a função para recriar o heap} \\
        \CommentSty{// no vetor reduzido}\\
        ConstroiHeap(V, 0, i)\\
    }
}
\end{algorithm}
}
\end{frame}

%%%%%%%%%%%%%%%%%%%%%%%%%%%%%%%%%%%%%%%%%%%%%%%%%%%%%%%%%%%%%%%%%%%%%%%%%%%%%%%%%%%%%%%%%%%%%%%%%%%%%%%%%%%%%

\begin{frame}[fragile]{Exercício 1}{Solução}
\begin{lstlisting}[style=CStyle]
void heapsort(int n, int *vetor) {
    int i, aux;
    for(i = (n/2)-1; i >= 0; i--) {
        constroi_heap(vetor, i, n);
    }
    for(i = n-1; i >= 0; i--) {
        aux = vetor[0];
        vetor[0] = vetor[i];
        vetor[i] = aux;
        constroi_heap(vetor, 0, i);
    }
}
\end{lstlisting}  
\end{frame}

%%%%%%%%%%%%%%%%%%%%%%%%%%%%%%%%%%%%%%%%%%%%%%%%%%%%%%%%%%%%%%%%%%%%%%%%%%%%%%%%%%%%%%%%%%%%%%%%%%%%%%%%%%%%%%
\section{Exercício 2}
%%%%%%%%%%%%%%%%%%%%%%%%%%%%%%%%%%%%%%%%%%%%%%%%%%%%%%%%%%%%%%%%%%%%%%%%%%%%%%%%%%%%%%%%%%%%%%%%%%%%%%%%%%%%%

\begin{frame}[fragile]{Exercício 2}{Tempo Processamento}
\begin{itemize}
\item Calcule o o tempo de processamento para ordenar cada um dos vetores do exercício anterior utilizando o algoritmo {\bf Heapsort}. 
\item Para isso, utilize a função \verb|clock()|. Acesse \href{http://wurthmann.blogspot.com/2015/04/medir-tempo-de-execucao-em-c.html}{\color{blue} esse link} para entender o funcionamento da função \verb|clock()|.
\end{itemize}
\end{frame}

%%%%%%%%%%%%%%%%%%%%%%%%%%%%%%%%%%%%%%%%%%%%%%%%%%%%%%%%%%%%%%%%%%%%%%%%%%%%%%%%%%%%%%%%%%%%%%%%%%%%%%%%%%%%%

\begin{frame}[fragile]{Exercício 2}{Tempo Processamento}
\begin{itemize}
\item O código a seguir apresenta um exemplo de ordenação utilizando um vetor com valores aleatórios. 
\item Modifique esse código a fim de obter o tempo de processamento.
\end{itemize}
\end{frame}


%%%%%%%%%%%%%%%%%%%%%%%%%%%%%%%%%%%%%%%%%%%%%%%%%%%%%%%%%%%%%%%%%%%%%%%%%%%%%%%%%%%%%%%%%%%%%%%%%%%%%%%%%%%%%

\begin{frame}[fragile]{Exercício 2}{Solução}
\begin{lstlisting}[style=CStyle]
#define tamanho_vetor 100000
#define valor_max 1000000
#define valor_min 0
int main() {
    int i, *vetor;
    srand(0);
    vetor = malloc(tamanho_vetor*sizeof(int));
    for (i = 0; i < tamanho_vetor; i++)
        vetor[i] = (rand() % valor_max) + valor_min;
    for(i = 0; i < tamanho_vetor; i++)
        printf("%d ", vetor[i]);
    heapsort(tamanho_vetor, vetor);
    printf("\n");
    for(i = 0; i < tamanho_vetor; i++)
        printf("%d ", vetor[i]);
    free (vetor);
    return 0;
}
\end{lstlisting}  
\end{frame}


%%%%%%%%%%%%%%%%%%%%%%%%%%%%%%%%%%%%%%%%%%%%%%%%%%%%%%%%%%%%%%%%%%%%%%%%%%%%%%%%%%%%%%%%%%%%%%%%%%%%%%%%%%%%%
\section{Exercício 3}
%%%%%%%%%%%%%%%%%%%%%%%%%%%%%%%%%%%%%%%%%%%%%%%%%%%%%%%%%%%%%%%%%%%%%%%%%%%%%%%%%%%%%%%%%%%%%%%%%%%%%%%%%%%%%

\begin{frame}
\frametitle{Exercício 3}
Utilizando o algoritmo de ordenação {\bf Quicksort}, ordene um vetor composto por 1.000.000, 2.000.000 e 3.000.000 valores. Para cada um, execute os seguintes testes:
\begin{itemize}
 \item vetor composto por números aleatórios;
 \item vetor composto por números em ordem crescente;
 \item vetor composto por números em ordem decrescente.
\end{itemize} 
\end{frame}

%%%%%%%%%%%%%%%%%%%%%%%%%%%%%%%%%%%%%%%%%%%%%%%%%%%%%%%%%%%%%%%%%%%%%%%%%%%%%%%%%%%%%%%%%%%%%%%%%%%%%%%%%%%%%

\begin{frame}{Exercício 3}{Pseudo-código}
\scalebox{0.9}{
\begin{algorithm}[H]
\caption{QuicksortOrdena} 
\label{QuicksortOrdena}
\Entrada{Vetor $V[0..n-1]$, tamanho $n$}
\Saida{Vetor $V$ ordenado}
\Inicio{
	Quicksort($V, 0, n-1$) \\
}
\end{algorithm}
}

\scalebox{0.9}{
\begin{algorithm}[H]
\caption{Quicksort} 
\label{Quicksort}
\Entrada{Vetor $V[Esq..Dir]$, Esq, Dir}
\Saida{Vetor $V$ ordenado}
\Inicio{
	$(i, j) \leftarrow Particiona(V, Esq, Dir)$ \\
	\Se {( $Esq < j$ ) } {
        Quicksort($V, Esq, j$ ) \\
      }
    	\Se {( $i < Dir$ )} {
        Quicksort($V, i , Dir$) \\
	}
}
\end{algorithm}
}
\end{frame}

%%%%%%%%%%%%%%%%%%%%%%%%%%%%%%%%%%%%%%%%%%%%%%%%%%%%%%%%%%%%%%%%%%%%%%%%%%%%%%%%%%%%%%%%%%%%%%%%%%%%%%%%%%%%%

\begin{frame}{Exercício 3}{Pseudo-código}
\scalebox{0.8}{
\begin{algorithm}[H]
\caption{Particiona} 
\label{Particiona}
\Entrada{Vetor $V[Esq..Dir]$, Esq, Dir}
\Saida{Vetor $V$ ordenado}
\Inicio{
    $i \leftarrow Esq$; $j \leftarrow Dir$\\
    $x \leftarrow V[\frac{(i + j )}{2} ]$ \\
    \Repita{($i > j$)} { 
        \Enqto {( $x > V[ i ]$ )} { 
            $i \leftarrow i + 1$\\
        }
        \Enqto {( $x < V[ j ]$ )} { 
            $j \leftarrow j - 1$\\
        }
        \Se {( $i \leq j$ ) }
        { 
           Trocar $V[ i ] \leftrightarrow V[ j ] $\\ 
		$i \leftarrow i + 1$\\
		$j \leftarrow j - 1$\\
        }
    } 
    \Retorna $(i, j)$
}
\end{algorithm}
}
\end{frame}

%%%%%%%%%%%%%%%%%%%%%%%%%%%%%%%%%%%%%%%%%%%%%%%%%%%%%%%%%%%%%%%%%%%%%%%%%%%%%%%%%%%%%%%%%%%%%%%%%%%%%%%%%%%%%

\begin{frame}[fragile]{Exercício 3}{Solução}
\begin{lstlisting}[style=CStyle,basicstyle=\tiny]
void quicksort(int *a, int left, int right) {
    int i, j, x, y;
    i = left; j = right;
    x = a[(left + right) / 2];
    while(i <= j) {
        while(a[i] < x && i < right)
            i++;
        while(a[j] > x && j > left) 
            j--;
        if(i <= j) {
            y = a[i];
            a[i] = a[j];
            a[j] = y;
            i++; j--;
        }
    }
    if(j > left) 
        quicksort(a, left, j);

    if(i < right) 
        quicksort(a, i, right);
}
\end{lstlisting}  
\end{frame}

%%%%%%%%%%%%%%%%%%%%%%%%%%%%%%%%%%%%%%%%%%%%%%%%%%%%%%%%%%%%%%%%%%%%%%%%%%%%%%%%%%%%%%%%%%%%%%%%%%%%%%%%%%%%%
\section{Exercício 4}
%%%%%%%%%%%%%%%%%%%%%%%%%%%%%%%%%%%%%%%%%%%%%%%%%%%%%%%%%%%%%%%%%%%%%%%%%%%%%%%%%%%%%%%%%%%%%%%%%%%%%%%%%%%%%

\begin{frame}[fragile]{Exercício 4}{Tempo Processamento}
Calcule o tempo de processamento para ordenar cada um dos vetores do exercício anterior utilizando o algoritmo {\bf Quicksort}. 
\end{frame}
%

%%%%%%%%%%%%%%%%%%%%%%%%%%%%%%%%%%%%%%%%%%%%%%%%%%%%%%%%%%%%%%%%%%%%%%%%%%%%%%%%%%%%%%%%%%%%%%%%%%%%%%%%%%%%%

\begin{frame}[plain]
  \titlepage
\end{frame}

\end{document}
