\documentclass[aspectratio=169]{beamer}
\usetheme{Boadilla}
%\usetheme{Warsaw}
%\setbeamercovered{transparent}
\beamertemplatetransparentcoveredhigh
\usepackage[portuges]{babel}
\usepackage[utf8]{inputenc}
\usepackage{lmodern}
\usepackage[T1]{fontenc}
\usepackage[portuguese, linesnumbered, vlined, titlenumbered, ruled]{algorithm2e}
\usepackage{hyperref} 

\usepackage{xcolor}

\definecolor{codegreen}{rgb}{0,0.6,0}
\definecolor{codegray}{rgb}{0.5,0.5,0.5}
\definecolor{codepurple}{rgb}{0.58,0,0.82}
\definecolor{backcolour}{rgb}{0.95,0.95,0.92}

\usepackage{listings}
\lstdefinestyle{CStyle}{
    language=C++,
    backgroundcolor=\color{backcolour},   
    commentstyle=\color{codegreen},
    keywordstyle=\color{magenta},
    numberstyle=\tiny\color{codegray},
    stringstyle=\color{codepurple},
    basicstyle=\ttfamily\footnotesize,
    breakatwhitespace=false,         
    breaklines=true,                 
    keepspaces=true,                 
    numbers=left,       
    numbersep=5pt,                  
    showspaces=false,                
    showstringspaces=false,
    showtabs=false,                  
    tabsize=2,
}


\newcommand{\eng}[1]{\textsl{#1}}
\newcommand{\cod}[1]{\texttt{#1}}

\title[Aula Prática Algoritmos de Ordenação]{Busca Binária\\
   Algoritmos e Estrutura de Dados}
\author[Frederico Santos de Oliveira]{prof. Frederico Santos de Oliveira}
\institute[UFMT]{Universidade Federal de Mato Grosso\\ Instituto de Engenharia}
\date{}
\begin{document}

\begin{frame}[plain]
  \titlepage
\end{frame}

%\section*{Roteiro}

\begin{frame}
  \frametitle{Agenda}
  \tableofcontents
\end{frame}


\section{Exercício 1}

\begin{frame}
\frametitle{Exercício 1}

\begin{itemize}
 \item Crie um vetor composto de 1.000.000, 2.000.000 e 3.000.000 elementos ordenados em ordem crescente. 
 \item Implemente a busca binária iterativa e calcule quantas comparacões são necessárias para encontrar um elmento.
\end{itemize} 
\end{frame}


%------------------------------------------------
%------------------------------------------------

\begin{frame}
\frametitle{Pseudo-código pesquisa binária (Iterativo)}
\scalebox{0.8}{
\begin{algorithm}[H]
\caption{BuscaBináriaIterativa($V,x,i,f$)} 
\label{pesquisa_binaria_iterativo}
\Entrada{Vetor $V$, chave $x$ de busca, índices inicial $i$ e final $f$ de $V$.}
\Saida{Índice $m$ da posição de $x$ em $V$ ou determina que $x \notin V$.}
\Inicio{
  $encontrado \leftarrow FALSE$ \\
  \Enqto{($i \leq f) \textrm{ AND } (\textrm{ NOT }(encontrado$))}  {
    $m \leftarrow \lfloor \frac{i+f}{2} \rfloor$ \\
    \Se{($V[m] = x$)}{
	$encontrado \leftarrow TRUE$ \\
    }
    \SenaoSe{($x < V[m]$)} {
	$f \leftarrow m - 1$ \\
    } \Senao {
	$i \leftarrow m + 1$ \\
    }    
    
  }
  \Se{($encontrado$)} {
    \Retorna $m$
  } \Senao {
    \Retorna \textrm{``Não encontrado''}
  }
}
\end{algorithm}
}
\end{frame}


%%%%%%%%%%%%%%%%%%%%%%%%%%%%%%%%%%%%%%%%%%%%%%%%%%%%%%%%%%%%%%%%%%%%%%%%%%%%%%%%%%%%%%%%%%%%%%%%%%%%%%%%%%%%%

\begin{frame}[fragile]{Exercício 1}{Solução}
\begin{lstlisting}[style=CStyle]
int iterative_binary_search(int *vetor, int inicio, int fim, int valor) {
    int encontrado = 0, meio = 0;
    while ((inicio < fim) && !(encontrado)) {
        meio = (inicio + fim) / 2;
        if (vetor[meio] == valor) 
            encontrado = 1;
        else 
            if (valor < vetor[meio]) 
                fim = meio - 1;
            else 
                inicio = meio + 1;

    }
    if (encontrado)
        return meio;
    else
        return -1;

}
\end{lstlisting}  
\end{frame}

\section{Exercício 2}

\begin{frame}
\frametitle{Exercício 2}
Implemente a versão recursiva do algoritmo de busca binária.
\end{frame}

%------------------------------------------------

\begin{frame}
\frametitle{Pseudo-código pesquisa binária (Recursivo)}
\begin{algorithm}[H]
\caption{BuscaBinária($V,x,i,f$)} 
\label{pesquisa_binaria_recursiva}
\Entrada{Vetor $V$, chave $x$ de busca, índices inicial $i$ e final $f$ de $V$.}
\Saida{Índice $i$ da posição de $x$ em $V$ ou determina que $x \notin V$.}
\Inicio{
  \Se{($i = f$)} {
    \Se{($V[i] = x$)} {
	\Retorna i
      } 
      \Senao {
	\Retorna \textrm{``Não encontrado''}
      }
  }
  $m \leftarrow \lfloor \frac{i+f}{2} \rfloor$ \\
  \Se{($V[m] > x$)} {
      \Retorna BuscaBinária($V,x,i,m-1$)      
  }
  \Senao {
      \Retorna BuscaBinária($V,x,m,f$)
  }
}
\end{algorithm}
\end{frame}



%%%%%%%%%%%%%%%%%%%%%%%%%%%%%%%%%%%%%%%%%%%%%%%%%%%%%%%%%%%%%%%%%%%%%%%%%%%%%%%%%%%%%%%%%%%%%%%%%%%%%%%%%%%%%

\begin{frame}[fragile]{Exercício 2}{Solução}
\begin{lstlisting}[style=CStyle]
int recursive_binary_search(int *vetor, int inicio, int fim, int valor) {
    int meio = (inicio + fim) / 2;
    if (inicio == fim)
            return -1;
    if (vetor[meio] == valor)
        return meio;
    else {
        if (vetor[meio] > valor)
            recursive_binary_search(vetor, inicio, meio-1, valor);
        else
            recursive_binary_search(vetor, meio+1, fim, valor);
    }
}
\end{lstlisting}  
\end{frame}



\begin{frame}
  \frametitle{FIM}
\begin{itemize}
\item \alert{FIM}
\end{itemize}
\end{frame}	


\end{document}
