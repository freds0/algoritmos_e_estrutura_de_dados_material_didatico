\documentclass{beamer}
\usetheme{Boadilla}
%\usetheme{Warsaw}
%\setbeamercovered{transparent}
\beamertemplatetransparentcoveredhigh
\usepackage[portuges]{babel}
\usepackage[utf8]{inputenc}
\usepackage{lmodern}
\usepackage[T1]{fontenc}
\usepackage[portuguese, linesnumbered, vlined, titlenumbered, ruled]{algorithm2e}
\usepackage{hyperref} 


\newcommand{\eng}[1]{\textsl{#1}}
\newcommand{\cod}[1]{\texttt{#1}}

\title[Aula Prática Algoritmos de Ordenação]{Countingsort\\
   Estrutura de Dados}
\author[Frederico Santos de Oliveira]{prof. Frederico Santos de Oliveira}
\institute[UFMT]{Universidade Federal de Mato Grosso\\ Instituto de Engenharia}
\date{}
\begin{document}

\begin{frame}[plain]
  \titlepage
\end{frame}

%\section*{Roteiro}

\begin{frame}
  \frametitle{Agenda}
  \tableofcontents
\end{frame}


\section{Exercício 1}

\begin{frame}
\frametitle{Exercício 1}
Considere o algoritmo de ordenação Countingsort apresentado em sala de aula. Utilizando alocação dinâmica, ordene um vetor composto de 1.000.000, 2.000.000 e 3.000.000 elementos. Para cada um, execute os seguintes testes:
\begin{itemize}
 \item vetor composto por números aleatórios;
 \item vetor composto por números em ordem crescente;
 \item vetor composto por números em ordem decrescente.
\end{itemize} 
Verifique quantas iteraçoes são necessárias em cada um dos testes. Analise os resultados.
\end{frame}


%------------------------------------------------

\begin{frame}{Countingsort}{Pseudo-código}
\scalebox{0.75}{
\begin{algorithm}[H]
\caption{Countingsort} 
\label{Countingsort}
\Entrada{Vetor $V[0..n]$, tamanho do vetor $n$}
\Saida{Vetor $V$ ordenado}
\Inicio{
  \CommentSty{// Considere o vetor auxiliar C[0..k] }\\
  \CommentSty{// onde k é o maior elemento presente no vetor.}\\
    \CommentSty{// Inicializa o vetor auxiliar com zeros.}\\
    \Para { ( $i \leftarrow 0$ até $n-1$ )} {
       $C[i] \leftarrow 0$ \\
    }
    \CommentSty{// Conta quantas vezes cada elemento aparece no vetor.}\\
    \Para {( $i\leftarrow 0$ até $n-1$) }
    {
        $C[V[i]] \leftarrow C[V[i]] + 1$\\
    }
    \CommentSty{// Insere no vetor original.}\\
    $j \leftarrow 0$ \\
    \Para {( $i\leftarrow 0$ até $n-1$) }
    {
	\Enqto {($C[i] > 0$)} {
	  $V[j] \leftarrow i$\\
	  $C[i] \leftarrow C[i] - 1$ \\
	  $j \leftarrow j + 1$\\
        }
    }
}
\end{algorithm}
}
\end{frame}

\section{Exercício 2}

\begin{frame}
\frametitle{Exercício 2}
Adapte o Countingsort de forma que ele ordene uma sequencia de letras do alfabeto.
\end{frame}


\begin{frame}
  \frametitle{FIM}
\begin{itemize}
\item \alert{FIM}
\end{itemize}
\end{frame}	


\end{document}
