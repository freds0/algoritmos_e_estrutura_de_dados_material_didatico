\documentclass[aspectratio=169]{beamer}
\usetheme{Boadilla}
%\usetheme{Warsaw}
%\setbeamercovered{transparent}
\beamertemplatetransparentcoveredhigh
\usepackage[portuges]{babel}
\usepackage[utf8]{inputenc}
\usepackage{lmodern}
\usepackage[T1]{fontenc}
\usepackage{hyperref} 
\usepackage[portuguese, linesnumbered, vlined, titlenumbered, ruled]{algorithm2e}
\SetKwRepeat{Registro}{registro \{}{\}}%
\newcommand{\Int}{\KwSty{int}}
\usepackage{listings}
\lstset{frame=tb,
  language=C,
  aboveskip=3mm,
  belowskip=3mm,
  showstringspaces=false,
  columns=flexible,
  basicstyle={\small\ttfamily},
  numbers=none,
  breaklines=true,
  breakatwhitespace=true,
  tabsize=3,
  numbers=left,
  stepnumber=1,
  xleftmargin=2em
}

\title[Aula Prática]{Algoritmos e Estrutura de Dados II}
\subtitle{Exercício Prático: Ponteiros e TADs}
\author[Frederico Santos de Oliveira]{prof. Frederico Santos de Oliveira}
\institute[UFMT]{Universidade Federal de Mato Grosso\\ Instituto de Engenharia}
\date{}

\begin{document}

\begin{frame}[plain]
  \titlepage
\end{frame}

%\section*{Roteiro}

\begin{frame}
  \frametitle{Agenda}
  \tableofcontents
\end{frame}

\section{Exercício 1}

\begin{frame}
\frametitle{Exercício 1}
Implemente um Tipo Abstrato de Dados (TAD) denominado Fracao para representar frações de números inteiros. Seu tipo abstrato deverá armazenar os dois elementos do tipo inteiro que formam uma fração: o numerador e o denominador. Para isso, crie o Tipo Abstrato de Dados conforme o pseudo-código a seguir:


\begin{algorithm}[H]
\caption{TAD Fracao} 
\label{pseudocodigo}
\Inicio{
 \Registro{fracao}{
    \Int{} num, den\; 
  }
}
\end{algorithm}
\end{frame}

\begin{frame}[fragile]
\frametitle{Exercício 1}
Na linguagem C, a TAD que representa uma fração tem a seguinte forma:

\color{blue}\rule{\linewidth}{4pt}
Código em C
\begin{lstlisting}
typedef struct {
  int num, den;
} fracao;
\end{lstlisting}
\end{frame}


\begin{frame}
\frametitle{Exercício 1}
Crie as seguintes funções para manipular o TAD fração:
\begin{enumerate}[a)]
 \item fracao *criarFracao(int n, int d);
 \begin{itemize}
  \item função que recebe dois numeros inteiros (numerador $n$ e denominador $d$) e retorna um ponteiro que aponta para a fração $\frac{n}{d}$;
 \end{itemize}
 \item void imprimirFracao(fracao *f); 
 \begin{itemize}
  \item função que recebe um ponteiro para uma fração e imprime usando o comando $printf(``\%d / \%d'', n, d);$
 \end{itemize}
 \item fracao *somaFracoes(fracao *f, fracao *g):
 \begin{itemize}
  \item função que recebe dois ponteiros para frações e retorna uma fração contendo o resultado da soma das duas frações.
 \end{itemize}
 \item fracao *multiplicarFracoes(fracao *f, fracao *g);
 \begin{itemize}
  \item função que recebe dois ponteiros para frações e retorna uma fração contendo o resultado da multiplicacao das duas frações.
 \end{itemize} 
 \item fracao *inverterFracao(fracao *f);
 \begin{itemize}
  \item função que recebe um ponteiro para uma fração $\frac{n}{d}$ e retorna uma fração inversa $\frac{d}{n}$.
 \end{itemize} 
\end{enumerate}

\end{frame}



\section{Exercício 2}

\begin{frame}
\frametitle{Exercício 2}
Implemente um TAD denominado Conjunto para representar conjuntos de números inteiros positivos. Seu tipo abstrato deverá armazenar os elementos do conjunto e o seu tamanho n. Considere que o tamanho máximo de um conjunto é 20 elementos e use arranjos de 1 dimensão (vetores) para a sua implementação. 
\end{frame}

\begin{frame}
\frametitle{Exercício 2}
Seu TAD deve possuir procedimentos (ou funções quando for o caso) para:
\begin{enumerate}[a)]
 \item criar um conjunto vazio;
 \item ler os dados de um conjunto;
 \item fazer a união de dois conjuntos;
 \item fazer a interseção de dois conjuntos;
 \item verificar se dois conjunto são iguais (possuem os mesmos elementos);
 \item imprimir um conjunto;
\end{enumerate}
Sugestão: Utilize um vetor de tamanho 20 para representar um conjunto. Dessa forma, a união de dois conjuntos poderá ter, no máximo, 40 elementos.
\end{frame}

\begin{frame}
  \frametitle{Fim}
\centering
\huge{Fim}
\end{frame}	


\end{document}
