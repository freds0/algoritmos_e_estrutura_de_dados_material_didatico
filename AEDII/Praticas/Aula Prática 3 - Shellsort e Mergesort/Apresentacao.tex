\documentclass[aspectratio=169]{beamer}
\usetheme{Boadilla}
%\usetheme{Warsaw}
%\setbeamercovered{transparent}
\beamertemplatetransparentcoveredhigh
\usepackage[portuges]{babel}
\usepackage[utf8]{inputenc}
\usepackage{lmodern}
\usepackage[T1]{fontenc}
\usepackage[portuguese, linesnumbered, vlined, titlenumbered, ruled]{algorithm2e}
\usepackage{hyperref} 

\usepackage{xcolor}

\definecolor{codegreen}{rgb}{0,0.6,0}
\definecolor{codegray}{rgb}{0.5,0.5,0.5}
\definecolor{codepurple}{rgb}{0.58,0,0.82}
\definecolor{backcolour}{rgb}{0.95,0.95,0.92}

\usepackage{listings}
\lstdefinestyle{CStyle}{
    language=C++,
    backgroundcolor=\color{backcolour},   
    commentstyle=\color{codegreen},
    keywordstyle=\color{magenta},
    numberstyle=\tiny\color{codegray},
    stringstyle=\color{codepurple},
    basicstyle=\ttfamily\footnotesize,
    breakatwhitespace=false,         
    breaklines=true,                 
    keepspaces=true,                 
    numbers=left,       
    numbersep=5pt,                  
    showspaces=false,                
    showstringspaces=false,
    showtabs=false,                  
    tabsize=2,
}


\newcommand{\eng}[1]{\textsl{#1}}
\newcommand{\cod}[1]{\texttt{#1}}

\title[Algoritmos de Ordenação]{Algoritmos e Estrutura de Dados II}
\subtitle{Exercício Prático\\Shellsort e Mergesort}
\author[Frederico Santos de Oliveira]{prof. Frederico Santos de Oliveira}
\institute[UFMT]{Universidade Federal de Mato Grosso\\ Instituto de Engenharia}
\date{}


\AtBeginSection[]
{
	\begin{frame}
	\frametitle{Roteiro da Aula}
	\tableofcontents[currentsection]
\end{frame}
}


\begin{document}

%%%%%%%%%%%%%%%%%%%%%%%%%%%%%%%%%%%%%%%%%%%%%%%%%%%%%%%%%%%%%%%%%%%%%%%%%%%%%%%%%%%%%%%%%%%%%%%%%%%%%%%%%%%%%

\begin{frame}[plain]
  \titlepage
\end{frame}

%%%%%%%%%%%%%%%%%%%%%%%%%%%%%%%%%%%%%%%%%%%%%%%%%%%%%%%%%%%%%%%%%%%%%%%%%%%%%%%%%%%%%%%%%%%%%%%%%%%%%%%%%%%%%
%\section*{Roteiro}
%%%%%%%%%%%%%%%%%%%%%%%%%%%%%%%%%%%%%%%%%%%%%%%%%%%%%%%%%%%%%%%%%%%%%%%%%%%%%%%%%%%%%%%%%%%%%%%%%%%%%%%%%%%%%

\begin{frame}
  \frametitle{Agenda}
  \tableofcontents
\end{frame}

%%%%%%%%%%%%%%%%%%%%%%%%%%%%%%%%%%%%%%%%%%%%%%%%%%%%%%%%%%%%%%%%%%%%%%%%%%%%%%%%%%%%%%%%%%%%%%%%%%%%%%%%%%%%%
\section{Exercício 1}
%%%%%%%%%%%%%%%%%%%%%%%%%%%%%%%%%%%%%%%%%%%%%%%%%%%%%%%%%%%%%%%%%%%%%%%%%%%%%%%%%%%%%%%%%%%%%%%%%%%%%%%%%%%%%

\begin{frame}
\frametitle{Exercício 1}
Utilizando o algoritmo de ordenação {\bf Shellsort}, ordene um vetor formado por mil valores. Para cada um, execute os seguintes testes:
\begin{itemize}
 \item vetor composto por números aleatórios;
 \item vetor composto por números em ordem crescente;
 \item vetor composto por números em ordem decrescente.
\end{itemize} 
\end{frame}

%%%%%%%%%%%%%%%%%%%%%%%%%%%%%%%%%%%%%%%%%%%%%%%%%%%%%%%%%%%%%%%%%%%%%%%%%%%%%%%%%%%%%%%%%%%%%%%%%%%%%%%%%%%%%

\begin{frame}[fragile]{Exercício 1}{Pseudocódigo}
\scalebox{0.8}{
\begin{algorithm}[H]
\caption{Shellsort} 
\label{Shellsort}
\Entrada{Vetor $V[0..n-1]$, tamanho $n$}
\Saida{Vetor $V$ ordenado}
\Inicio{
  $h \leftarrow 1$ \\
  \Enqto{($h < n$)} {
  	$h \leftarrow 3h + 1$
  }
  \Enqto {($ h \geq 1$)} {
    $h \leftarrow \frac{h}{3}$ \\  
    \Para {($i \leftarrow h \textrm{ até } n - 1$)} {
      $chave \leftarrow V[ i ]$ \\
      $j \leftarrow i - h$ \\
      \Enqto {($j \geq 0 \textrm{ AND }  V[ j ] > chave$)} {
  	  $V[ j + h ] \leftarrow V[ j ]$ \\
	  $j \leftarrow j - h$ \\
      }
      $V[ j + h ] \leftarrow chave$ \\
    }
  }
}
\end{algorithm}
}
\end{frame}

%%%%%%%%%%%%%%%%%%%%%%%%%%%%%%%%%%%%%%%%%%%%%%%%%%%%%%%%%%%%%%%%%%%%%%%%%%%%%%%%%%%%%%%%%%%%%%%%%%%%%%%%%%%%%

\begin{frame}[fragile]{Exercício 1}{Solução}
\begin{lstlisting}[style=CStyle]
void shellsort(int n, int *vetor) {
    int h, i, j, chave;
    h = 1;
    while (h < n) 
        h = 3 * h + 1;
    while(h>=1) {
        h=h/3;
        for(i = h; i < n; i++) {
            chave = vetor[i];
            j = i - h;
            while(j>=0 && vetor[j]>chave) {
                vetor[j+h] = vetor[j];
                j = j - h;
            }
            vetor[j+h] = chave;
        }
    }
} 
\end{lstlisting}  
\end{frame}

%%%%%%%%%%%%%%%%%%%%%%%%%%%%%%%%%%%%%%%%%%%%%%%%%%%%%%%%%%%%%%%%%%%%%%%%%%%%%%%%%%%%%%%%%%%%%%%%%%%%%%%%%%%%%
\section{Exercício 2}
%%%%%%%%%%%%%%%%%%%%%%%%%%%%%%%%%%%%%%%%%%%%%%%%%%%%%%%%%%%%%%%%%%%%%%%%%%%%%%%%%%%%%%%%%%%%%%%%%%%%%%%%%%%%%

\begin{frame}[fragile]{Exercício 2}{Tempo Processamento}
\begin{itemize}
\item Calcule o tempo de processamento para ordenar cada um dos vetores do exercício anterior utilizando o algoritmo {\bf Shellsort}. 
\item Para isso, utilize a função \verb|clock()|. Acesse \href{http://wurthmann.blogspot.com/2015/04/medir-tempo-de-execucao-em-c.html}{\color{blue} esse link} para entender o funcionamento da função \verb|clock()|.
\end{itemize}
\end{frame}

%%%%%%%%%%%%%%%%%%%%%%%%%%%%%%%%%%%%%%%%%%%%%%%%%%%%%%%%%%%%%%%%%%%%%%%%%%%%%%%%%%%%%%%%%%%%%%%%%%%%%%%%%%%%%

\begin{frame}[fragile]{Exercício 2}{Tempo Processamento}
\begin{itemize}
\item O código a seguir apresenta um exemplo de ordenação utilizando um vetor com valores aleatórios. 
\item Modifique esse código a fim de obter o tempo de processamento.
\end{itemize}
\end{frame}


%%%%%%%%%%%%%%%%%%%%%%%%%%%%%%%%%%%%%%%%%%%%%%%%%%%%%%%%%%%%%%%%%%%%%%%%%%%%%%%%%%%%%%%%%%%%%%%%%%%%%%%%%%%%%

\begin{frame}[fragile]{Exercício 2}{Solução}
\begin{lstlisting}[style=CStyle]
#define tamanho_vetor 1000
#define valor_max 1000
#define valor_min 0
int main() {
    int i, *vetor;
    srand(0);
    vetor = malloc(tamanho_vetor*sizeof(int));
    for (i = 0; i < tamanho_vetor; i++) 
        vetor[i] = (rand() % valor_max) + valor_min;
    for(i = 0; i < tamanho_vetor; i++)
        printf("%d ", vetor[i]);
    shellsort(tamanho_vetor, vetor);
    printf("\n");
    for(i = 0; i < tamanho_vetor; i++)
        printf("%d ", vetor[i]);
    free (vetor);
    return 0;
}
\end{lstlisting}  
\end{frame}


%%%%%%%%%%%%%%%%%%%%%%%%%%%%%%%%%%%%%%%%%%%%%%%%%%%%%%%%%%%%%%%%%%%%%%%%%%%%%%%%%%%%%%%%%%%%%%%%%%%%%%%%%%%%%
\section{Exercício 3}
%%%%%%%%%%%%%%%%%%%%%%%%%%%%%%%%%%%%%%%%%%%%%%%%%%%%%%%%%%%%%%%%%%%%%%%%%%%%%%%%%%%%%%%%%%%%%%%%%%%%%%%%%%%%%

\begin{frame}
\frametitle{Exercício 3}
Utilizando o algoritmo de ordenação {\bf Mergesort}, ordene um vetor formado por mil valores. Para cada um, execute os seguintes testes:
\begin{itemize}
 \item vetor composto por números aleatórios;
 \item vetor composto por números em ordem crescente;
 \item vetor composto por números em ordem decrescente.
\end{itemize} 
\end{frame}

%%%%%%%%%%%%%%%%%%%%%%%%%%%%%%%%%%%%%%%%%%%%%%%%%%%%%%%%%%%%%%%%%%%%%%%%%%%%%%%%%%%%%%%%%%%%%%%%%%%%%%%%%%%%%

\begin{frame}{Exercício 3}{Pseudo-código}

\scalebox{0.8}{
\begin{algorithm}[H]
\caption{MergesortOrdena} 
\label{Mergesort}
\Entrada{Vetor $V[0..n-1]$, tamanho do vetor $n$. }
\Saida{Vetor $V$ ordenado}
\Inicio{
    Mergesort(V,0,n)\\
}
\end{algorithm}
}
\scalebox{0.8}{
\begin{algorithm}[H]
\caption{Mergesort} 
\label{Mergesort}
\Entrada{Vetor $V[i..f-1]$, início $i$ de $V$, e o final $f$ de $V$. }
\Saida{Vetor $V$ ordenado}
\Inicio{
  \Se {($ i < f -1 $)} {
    $m \leftarrow \frac{(i + f)}{2} $ \\  
    Mergesort(V,i,m)\\
    Mergesort(V,m,f)\\
    Merge(V, i, m, f) \\
  }
}
\end{algorithm}
} 
%\\
% \Tiny{Adaptado de \citeonline{Ziviani2011}.}
\end{frame}

%%%%%%%%%%%%%%%%%%%%%%%%%%%%%%%%%%%%%%%%%%%%%%%%%%%%%%%%%%%%%%%%%%%%%%%%%%%%%%%%%%%%%%%%%%%%%%%%%%%%%%%%%%%%%

\begin{frame}[fragile]{Exercício 3}{Solução}
\begin{lstlisting}[style=CStyle]
void mergesort(int *vetor, int i, int f) {
    int m;
    if(i<f-1) {
        m = (i+f)/2;
        mergesort(vetor, i, m);
        mergesort(vetor, m, f);
        merge(vetor, i, m, f);
    }
}
void mergesort_ordena(int n, int *vetor) {
    mergesort(vetor, 0, n);
}
\end{lstlisting}  
\end{frame}

%%%%%%%%%%%%%%%%%%%%%%%%%%%%%%%%%%%%%%%%%%%%%%%%%%%%%%%%%%%%%%%%%%%%%%%%%%%%%%%%%%%%%%%%%%%%%%%%%%%%%%%%%%%%%

\begin{frame}{Exercício 3}{Pseudo-código}
\scalebox{0.62}{
\begin{algorithm}[H]
\caption{Merge} 
\label{Merge}
\Entrada{Vetor $V[ini..fim-1]$, $ini$, $meio$, $fim$. }
\Saida{Vetor $V$ ordenado}
\Inicio{
      \CommentSty{// Considere o vetor auxiliar W[ini..fim-1]}\\
	$i \leftarrow ini$; 
	$j \leftarrow meio$;
	$k \leftarrow 0$\\
	\Enqto {( $i < meio$ e $j < fim$)} {
		\Se{$V[i] \leq V[j]$} {
			$W[k] \leftarrow V[i]$ \\
			$i \leftarrow i + 1$\\
		}
		\Senao{
			$W[k] \leftarrow V[j]$ \\
			$j \leftarrow j + 1$;\\
		}
		$k \leftarrow k + 1$\\
	}
	\Enqto {($i<meio$)}{
		$W[k] \leftarrow V[i]$\\
		$i \leftarrow i + 1$; $k \leftarrow k + 1$\\
	}
	\Enqto {($j<fim$)}{
		$W[k] \leftarrow V[j]$\\
		$j \leftarrow j + 1$; $k \leftarrow k + 1$\\
	}	
	\Para {($i \leftarrow ini \textrm{ até } fim - 1 $)} {
		$V[i] \leftarrow W[i- ini]$\\
	}
}
\end{algorithm}
} % scalebox
\end{frame}

%%%%%%%%%%%%%%%%%%%%%%%%%%%%%%%%%%%%%%%%%%%%%%%%%%%%%%%%%%%%%%%%%%%%%%%%%%%%%%%%%%%%%%%%%%%%%%%%%%%%%%%%%%%%%

\begin{frame}[fragile]{Exercício 3}{Solução}
\begin{lstlisting}[style=CStyle,basicstyle=\tiny]
void merge(int *vetor, int ini, int meio, int fim) {
    int i = ini, j = meio, k = 0, *w = malloc(fim*sizeof(int));
    while(i < meio && j < fim) {
        if(vetor[i] <= vetor[j]) {
            w[k] = vetor[i];
            i++;
        }
        else {
            w[k] = vetor[j];
            j++;
        }
        k++;
    }
    while(i < meio) {
        w[k] = vetor[i];
        i++;k++;
    }
    while(j < fim) {
        w[k] = vetor[j];
        j++; k++;
    }
    for(i = ini; i < fim; i++) {
        vetor[i] = w[i-ini];
    }
    free(w);
}
\end{lstlisting}  
\end{frame}

%%%%%%%%%%%%%%%%%%%%%%%%%%%%%%%%%%%%%%%%%%%%%%%%%%%%%%%%%%%%%%%%%%%%%%%%%%%%%%%%%%%%%%%%%%%%%%%%%%%%%%%%%%%%%
\section{Exercício 4}
%%%%%%%%%%%%%%%%%%%%%%%%%%%%%%%%%%%%%%%%%%%%%%%%%%%%%%%%%%%%%%%%%%%%%%%%%%%%%%%%%%%%%%%%%%%%%%%%%%%%%%%%%%%%%

\begin{frame}[fragile]{Exercício 4}{Tempo Processamento}
Calcule o tempo de processamento para ordenar cada um dos vetores do exercício anterior utilizando o algoritmo {\bf Mergesort}. 
\end{frame}


%%%%%%%%%%%%%%%%%%%%%%%%%%%%%%%%%%%%%%%%%%%%%%%%%%%%%%%%%%%%%%%%%%%%%%%%%%%%%%%%%%%%%%%%%%%%%%%%%%%%%%%%%%%%%

\begin{frame}[plain]
  \titlepage
\end{frame}

\end{document}
