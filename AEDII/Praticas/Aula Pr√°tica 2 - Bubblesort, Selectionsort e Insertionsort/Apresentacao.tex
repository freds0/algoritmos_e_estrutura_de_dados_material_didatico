\documentclass[aspectratio=169]{beamer}
\usetheme{Boadilla}
%\usetheme{Warsaw}
%\setbeamercovered{transparent}
\beamertemplatetransparentcoveredhigh
\usepackage[portuges]{babel}
\usepackage[utf8]{inputenc}
\usepackage{lmodern}
\usepackage[T1]{fontenc}
\usepackage[portuguese, linesnumbered, vlined, titlenumbered, ruled]{algorithm2e}
\usepackage{hyperref} 

\usepackage{xcolor}

\definecolor{codegreen}{rgb}{0,0.6,0}
\definecolor{codegray}{rgb}{0.5,0.5,0.5}
\definecolor{codepurple}{rgb}{0.58,0,0.82}
\definecolor{backcolour}{rgb}{0.95,0.95,0.92}

\usepackage{listings}
\lstdefinestyle{CStyle}{
    language=C++,
    backgroundcolor=\color{backcolour},   
    commentstyle=\color{codegreen},
    keywordstyle=\color{magenta},
    numberstyle=\tiny\color{codegray},
    stringstyle=\color{codepurple},
    basicstyle=\ttfamily\footnotesize,
    breakatwhitespace=false,         
    breaklines=true,                 
    keepspaces=true,                 
    numbers=left,       
    numbersep=5pt,                  
    showspaces=false,                
    showstringspaces=false,
    showtabs=false,                  
    tabsize=2,
}


\newcommand{\eng}[1]{\textsl{#1}}
\newcommand{\cod}[1]{\texttt{#1}}

\title[Algoritmos de Ordenação]{Algoritmos e Estrutura de Dados II}
\subtitle{Exercício Prático\\Bubblesort, Insertionsort e Selectionsort}
\author[Frederico Santos de Oliveira]{prof. Frederico Santos de Oliveira}
\institute[UFMT]{Universidade Federal de Mato Grosso\\ Instituto de Engenharia}
\date{}


\AtBeginSection[]
{
	\begin{frame}
	\frametitle{Roteiro da Aula}
	\tableofcontents[currentsection]
\end{frame}
}


\begin{document}

%%%%%%%%%%%%%%%%%%%%%%%%%%%%%%%%%%%%%%%%%%%%%%%%%%%%%%%%%%%%%%%%%%%%%%%%%%%%%%%%%%%%%%%%%%%%%%%%%%%%%%%%%%%%%

\begin{frame}[plain]
  \titlepage
\end{frame}

%%%%%%%%%%%%%%%%%%%%%%%%%%%%%%%%%%%%%%%%%%%%%%%%%%%%%%%%%%%%%%%%%%%%%%%%%%%%%%%%%%%%%%%%%%%%%%%%%%%%%%%%%%%%%
%\section*{Roteiro}
%%%%%%%%%%%%%%%%%%%%%%%%%%%%%%%%%%%%%%%%%%%%%%%%%%%%%%%%%%%%%%%%%%%%%%%%%%%%%%%%%%%%%%%%%%%%%%%%%%%%%%%%%%%%%

\begin{frame}
  \frametitle{Agenda}
  \tableofcontents
\end{frame}

%%%%%%%%%%%%%%%%%%%%%%%%%%%%%%%%%%%%%%%%%%%%%%%%%%%%%%%%%%%%%%%%%%%%%%%%%%%%%%%%%%%%%%%%%%%%%%%%%%%%%%%%%%%%%
\section{Exercício 1}
%%%%%%%%%%%%%%%%%%%%%%%%%%%%%%%%%%%%%%%%%%%%%%%%%%%%%%%%%%%%%%%%%%%%%%%%%%%%%%%%%%%%%%%%%%%%%%%%%%%%%%%%%%%%%

\begin{frame}{Exercício 1}
Desenvolva um programa que gere um vetor com valores aleatórios.
\end{frame}


\begin{frame}[fragile]{Exercício 1}
Primeiramente, vamos definir algumas constantes: 
\begin{itemize}
\item \verb|tamanho_vetor|: define a quantidade de elementos no vetor.
\item \verb|valor_max|: define o maior valor a ser inserido no vetor.
\item \verb|valor_min|: define o menor valor a ser inserido no vetor.
\end{itemize}
Definindo as constantes, respectivamente, com os valores 10, 20 e 0, o código na linguagem C fica assim:

\begin{lstlisting}[style=CStyle]
#define tamanho_vetor 10
#define valor_max 20
#define valor_min 0
\end{lstlisting}     
\end{frame}

%%%%%%%%%%%%%%%%%%%%%%%%%%%%%%%%%%%%%%%%%%%%%%%%%%%%%%%%%%%%%%%%%%%%%%%%%%%%%%%%%%%%%%%%%%%%%%%%%%%%%%%%%%%%%

\begin{frame}[fragile]{Exercício 1}
Agora, dentro da função \verb|main()| vamos alocar um vetor de inteiros de tamanho igual \verb|tamanho_vetor|:
\begin{lstlisting}[style=CStyle]
int *vetor;
vetor = malloc(tamanho_vetor*sizeof(int));
\end{lstlisting}     
Não se esqueça de desalocar no final:
\begin{lstlisting}[style=CStyle]
free(vetor);
\end{lstlisting}     
\end{frame}

%%%%%%%%%%%%%%%%%%%%%%%%%%%%%%%%%%%%%%%%%%%%%%%%%%%%%%%%%%%%%%%%%%%%%%%%%%%%%%%%%%%%%%%%%%%%%%%%%%%%%%%%%%%%%

\begin{frame}[fragile]{Exercício 1}
\begin{itemize}
\item Para inserir números aleatórios no vetor utilizamos a função \verb|rand()|.
\item Acesse \href{http://linguagemc.com.br/valores-aleatorios-em-c-com-a-funcao-rand/}{\color{blue} esse link} para entender o funcionamento da função \verb|rand()| e \verb|srand()|.
\item A função \verb|rand()|  é frequentemente usada em conjunto com a função \verb|time(NULL)|, da seguinte forma: \verb|srand(time(NULL))|.
\end{itemize}
\end{frame}


%%%%%%%%%%%%%%%%%%%%%%%%%%%%%%%%%%%%%%%%%%%%%%%%%%%%%%%%%%%%%%%%%%%%%%%%%%%%%%%%%%%%%%%%%%%%%%%%%%%%%%%%%%%%%

\begin{frame}[fragile]{Exercício 1}{Vetor Aleatório}
Vamos inserir valores aleatórios no vetor:
\begin{lstlisting}[style=CStyle]
srand(0);
for (i = 0; i < tamanho_vetor; i++) 
    vetor[i] = (rand() % valor_max) + valor_min;

\end{lstlisting}  
Em seguida, vamor imprimí-los:
\begin{lstlisting}[style=CStyle]
for(i = 0; i < tamanho_vetor; i++)
    printf("%d ", vetor[i]);   
\end{lstlisting}  
Execute o código utilizando \verb|srand(time(NULL))|, lembre-se que é necessário incluir a biblioteca \verb|<time.h>|.
\end{frame}

%%%%%%%%%%%%%%%%%%%%%%%%%%%%%%%%%%%%%%%%%%%%%%%%%%%%%%%%%%%%%%%%%%%%%%%%%%%%%%%%%%%%%%%%%%%%%%%%%%%%%%%%%%%%%

\begin{frame}[fragile]{Exercício 1}{Solução}
O código completo fica:
\begin{lstlisting}[style=CStyle]
#include<stdio.h>
#include<stdlib.h>
#include<time.h>
#define tamanho_vetor 10
#define valor_max 20
#define valor_min 0
int main() {
    int i, *vetor;
    srand(time(NULL));
    vetor = malloc(tamanho_vetor*sizeof(int));
    for (i = 0; i < tamanho_vetor; i++) 
        vetor[i] = (rand() % valor_max) + valor_min;
    for(i = 0; i < tamanho_vetor; i++)
        printf("%d ", vetor[i]);
    free (vetor);
    return 0;
}        
\end{lstlisting}  

\end{frame}

%%%%%%%%%%%%%%%%%%%%%%%%%%%%%%%%%%%%%%%%%%%%%%%%%%%%%%%%%%%%%%%%%%%%%%%%%%%%%%%%%%%%%%%%%%%%%%%%%%%%%%%%%%%%%
\section{Exercício 2}
%%%%%%%%%%%%%%%%%%%%%%%%%%%%%%%%%%%%%%%%%%%%%%%%%%%%%%%%%%%%%%%%%%%%%%%%%%%%%%%%%%%%%%%%%%%%%%%%%%%%%%%%%%%%%

\begin{frame}[fragile]{Exercício 2}{Bubblesort}
Dado o pseudocódigo a seguir, implemente o algoritmo de ordenação {\bf Bubblesort} e execute-o em um vetor formado por valores aleatórios.
\end{frame}

%%%%%%%%%%%%%%%%%%%%%%%%%%%%%%%%%%%%%%%%%%%%%%%%%%%%%%%%%%%%%%%%%%%%%%%%%%%%%%%%%%%%%%%%%%%%%%%%%%%%%%%%%%%%%

\begin{frame}[fragile]{Exercício 2}{Pseudocódigo}
\begin{algorithm}[H]
\caption{Bubblesort} 
\label{Bubblesort}
\Entrada{Vetor $V[0..n-1]$, tamanho $n$}
\Saida{Vetor $V$ ordenado}
\Inicio{
  \Para {($i \leftarrow 1 \textrm{ até } n - 1$)} {
    \Para {($j \leftarrow 0 \textrm{ até } n - i - 1$)} {
    \Se { ($V[j] > V[j+1]$) } {
	Trocar $V[j] \leftrightarrow V[j+1]$
	}    
    }
  }
}
\end{algorithm}
\end{frame}

%%%%%%%%%%%%%%%%%%%%%%%%%%%%%%%%%%%%%%%%%%%%%%%%%%%%%%%%%%%%%%%%%%%%%%%%%%%%%%%%%%%%%%%%%%%%%%%%%%%%%%%%%%%%%
\begin{frame}[fragile]{Exercício 2}{Solução}
\begin{lstlisting}[style=CStyle]
void bubblesort(int n, int *vetor) {
    int i, j, aux;
    for(i=1; i<n; i++) {
        for(j=0; j<n-i; j++) {
            if(vetor[j]>vetor[j+1]) {
                aux = vetor[j+1];
                vetor[j+1] = vetor[j];
                vetor[j] = aux;
            }
        }
    }
}
\end{lstlisting}
\end{frame}

%%%%%%%%%%%%%%%%%%%%%%%%%%%%%%%%%%%%%%%%%%%%%%%%%%%%%%%%%%%%%%%%%%%%%%%%%%%%%%%%%%%%%%%%%%%%%%%%%%%%%%%%%%%%%
\section{Exercício 3}
%%%%%%%%%%%%%%%%%%%%%%%%%%%%%%%%%%%%%%%%%%%%%%%%%%%%%%%%%%%%%%%%%%%%%%%%%%%%%%%%%%%%%%%%%%%%%%%%%%%%%%%%%%%%%

\begin{frame}[fragile]{Exercício 3}{Selectionsort}
Dado o pseudocódigo a seguir, implemente o algoritmo de ordenação {\bf Selectionsort} e execute-o em um vetor formado por valores aleatórios.
\end{frame}

%%%%%%%%%%%%%%%%%%%%%%%%%%%%%%%%%%%%%%%%%%%%%%%%%%%%%%%%%%%%%%%%%%%%%%%%%%%%%%%%%%%%%%%%%%%%%%%%%%%%%%%%%%%%%

\begin{frame}[fragile]{Exercício 3}{Pseudocódigo}
\begin{algorithm}[H]
\caption{Selectionsort} 
\label{SelectionSort}
\Entrada{Vetor $V[0..n-1]$, tamanho $n$}
\Saida{Vetor $V$ ordenado}
\Inicio{
  \Para {$i \leftarrow 0 \textrm{ até } n-2$} {
    $min \leftarrow i$ \\
    \Para {$j \leftarrow i + 1 \textrm{ até } n-1$} {
	\Se{$V[j] < V[min]$} {
	  $min \leftarrow j$
	}
    }
    trocar $V[min] \leftrightarrow V[i]$ \\    
  }
}
\end{algorithm}
\end{frame}

%%%%%%%%%%%%%%%%%%%%%%%%%%%%%%%%%%%%%%%%%%%%%%%%%%%%%%%%%%%%%%%%%%%%%%%%%%%%%%%%%%%%%%%%%%%%%%%%%%%%%%%%%%%%%
\begin{frame}[fragile]{Exercício 3}{Solução}
\begin{lstlisting}[style=CStyle]
void selectionsort(int n, int *vetor) {
    int i, j, aux, min;
    for(i = 0; i < n-1; i++) {
        min = i;
        for(j = i+1; j < n; j++) {
            if(vetor[j] < vetor[min])  {
                min = j;
            }
        }
        aux = vetor[min];
        vetor[min] = vetor[i];
        vetor[i] = aux;
    }
}
\end{lstlisting}
\end{frame}

%%%%%%%%%%%%%%%%%%%%%%%%%%%%%%%%%%%%%%%%%%%%%%%%%%%%%%%%%%%%%%%%%%%%%%%%%%%%%%%%%%%%%%%%%%%%%%%%%%%%%%%%%%%%%
\section{Exercício 4}
%%%%%%%%%%%%%%%%%%%%%%%%%%%%%%%%%%%%%%%%%%%%%%%%%%%%%%%%%%%%%%%%%%%%%%%%%%%%%%%%%%%%%%%%%%%%%%%%%%%%%%%%%%%%%

\begin{frame}[fragile]{Exercício 4}{Inserionsort}
Dado o pseudocódigo a seguir, implemente o algoritmo de ordenação {\bf Inserionsort} e execute-o em um vetor formado por valores aleatórios.
\end{frame}

%%%%%%%%%%%%%%%%%%%%%%%%%%%%%%%%%%%%%%%%%%%%%%%%%%%%%%%%%%%%%%%%%%%%%%%%%%%%%%%%%%%%%%%%%%%%%%%%%%%%%%%%%%%%%

\begin{frame}[fragile]{Exercício 4}{Pseudocódigo}
\begin{algorithm}[H]
\caption{Insertionsort} 
\label{Insertionsort}
\Entrada{Vetor $V[0..n-1]$, tamanho $n$}
\Saida{Vetor $V$ ordenado}
\Inicio{
  \Para {($i \leftarrow 1 \textrm{ até } n - 1$)} {
    $chave \leftarrow V[ i ]$ \\
    $j \leftarrow i - 1$ \\
    \Enqto {($j \geq 0 \textrm{ AND }  V[ j ] > chave$)} {
	$V[j+1] \leftarrow V[j]$ \\
	$j \leftarrow j - 1$ \\
    }
    $V[j+1] \leftarrow chave$ \\
  }
}
\end{algorithm}
\end{frame}

%%%%%%%%%%%%%%%%%%%%%%%%%%%%%%%%%%%%%%%%%%%%%%%%%%%%%%%%%%%%%%%%%%%%%%%%%%%%%%%%%%%%%%%%%%%%%%%%%%%%%%%%%%%%%
\begin{frame}[fragile]{Exercício 4}{Solução}
\begin{lstlisting}[style=CStyle]
void insertionsort(int n, int *vetor) {
    int i, j, chave;
    for(i = 1; i < n; i++) {
        chave = vetor[i];
        j = i - 1;
        while(j>=0 && vetor[j]>chave) {
            vetor[j+1] = vetor[j];
            j = j - 1;
        }
        vetor[j+1] = chave;
    }
}
\end{lstlisting}
\end{frame}


%%%%%%%%%%%%%%%%%%%%%%%%%%%%%%%%%%%%%%%%%%%%%%%%%%%%%%%%%%%%%%%%%%%%%%%%%%%%%%%%%%%%%%%%%%%%%%%%%%%%%%%%%%%%%
\section{Exercício 5}
%%%%%%%%%%%%%%%%%%%%%%%%%%%%%%%%%%%%%%%%%%%%%%%%%%%%%%%%%%%%%%%%%%%%%%%%%%%%%%%%%%%%%%%%%%%%%%%%%%%%%%%%%%%%%

\begin{frame}[fragile]{Exercício 5}{Tempo Processamento}
\begin{itemize}
\item Calcule o tempo de processamento para ordenar um vetor contendo mil valores utilzando cada um dos algoritmos de ordenação. Para isso, utilizamos a função \verb|clock()|. 
\item Acesse \href{http://wurthmann.blogspot.com/2015/04/medir-tempo-de-execucao-em-c.html}{\color{blue} esse link} para entender o funcionamento da função \verb|clock()|.
\end{itemize}
\end{frame}

\begin{frame}{Introdução}
Analise o exemplo a seguir que demostra como deve-se medir o tempo de processamento de um determinado programa. 
\end{frame}

%%%%%%%%%%%%%%%%%%%%%%%%%%%%%%%%%%%%%%%%%%%%%%%%%%%%%%%%%%%%%%%%%%%%%%%%%%%%%%%%%%%%%%%%%%%%%%%%%%%%%%%%%%%%%

\begin{frame}[fragile]{Exercício 5}{Tempo Processamento}
\begin{lstlisting}[style=CStyle]
#include <time.h>
int main () {
    clock_t tempoInicial, tempoFinal;
    double tempoGasto;
    tempoInicial = clock(); // Obtem o clock inicial
    /* Adicione aqui a chamada do algoritmo de ordenacao */
    tempoFinal = clock(); // Obtem o clock final
    // Calcula o tempo gasto
    tempoGasto = (tempoFinal-tempoInicial)*1000 / CLOCKS_PER_SEC); 
    printf("Tempo em milisegundos: %lf", (double)tempoGasto);
}
\end{lstlisting}
\end{frame}

%%%%%%%%%%%%%%%%%%%%%%%%%%%%%%%%%%%%%%%%%%%%%%%%%%%%%%%%%%%%%%%%%%%%%%%%%%%%%%%%%%%%%%%%%%%%%%%%%%%%%%%%%%%%%
\section{Exercício 6}
%%%%%%%%%%%%%%%%%%%%%%%%%%%%%%%%%%%%%%%%%%%%%%%%%%%%%%%%%%%%%%%%%%%%%%%%%%%%%%%%%%%%%%%%%%%%%%%%%%%%%%%%%%%%%

\begin{frame}[fragile]{Exercício 6}{Ordenando um Registro}
\begin{itemize}
\item Execute o algoritmo {\bf Bubblesort}, entretanto, dessa vez deverá ordenar um vetor formado por registros, confome a estrutura de dados a seguir. 
\item Em seguida, calcule seu tempo de processamento para ordenacão de um vetor de dados aleatórios contendo mil valores.
\end{itemize}
\end{frame}

%%%%%%%%%%%%%%%%%%%%%%%%%%%%%%%%%%%%%%%%%%%%%%%%%%%%%%%%%%%%%%%%%%%%%%%%%%%%%%%%%%%%%%%%%%%%%%%%%%%%%%%%%%%%%

\begin{frame}[fragile]{Exercício 6}{Pseudocódigo}
\begin{algorithm}[H]
\caption{Selectionsort} 
\label{SelectionSort}
\Entrada{Vetor $V[0..n-1]$, tamanho $n$}
\Saida{Vetor $V$ ordenado}
\Inicio{
  \Para {$i \leftarrow 0 \textrm{ até } n-2$} {
    $min \leftarrow i$ \\
    \Para {$j \leftarrow i + 1 \textrm{ até } n-1$} {
	\Se{$V[j] < V[min]$} {
	  $min \leftarrow j$
	}
    }
    trocar $V[min] \leftrightarrow V[i]$ \\    
  }
}
\end{algorithm}
\end{frame}

%%%%%%%%%%%%%%%%%%%%%%%%%%%%%%%%%%%%%%%%%%%%%%%%%%%%%%%%%%%%%%%%%%%%%%%%%%%%%%%%%%%%%%%%%%%%%%%%%%%%%%%%%%%%%

\begin{frame}[fragile]{Exercício 6}{Estrutura de Dados}
\begin{lstlisting}[style=CStyle]
typedef struct {
    int chave;
    int valor;
} dado;

\end{lstlisting}
\end{frame}

%%%%%%%%%%%%%%%%%%%%%%%%%%%%%%%%%%%%%%%%%%%%%%%%%%%%%%%%%%%%%%%%%%%%%%%%%%%%%%%%%%%%%%%%%%%%%%%%%%%%%%%%%%%%%

\begin{frame}[fragile]{Exercício 6}{Ordenando um Registro}
O primeiro passo é criar um vetor de registros aleatórios.
\end{frame}

%%%%%%%%%%%%%%%%%%%%%%%%%%%%%%%%%%%%%%%%%%%%%%%%%%%%%%%%%%%%%%%%%%%%%%%%%%%%%%%%%%%%%%%%%%%%%%%%%%%%%%%%%%%%%

\begin{frame}[fragile]{Exercício 6}{Algoritmo de Ordenacão}
\begin{lstlisting}[style=CStyle]
#define tamanho_vetor 10
#define valor_max 20
#define valor_min 0
int main () {
    srand(0);
    vetor = malloc(tamanho_vetor*sizeof(dado));
    for (int i = 0; i < tamanho_vetor; i++) {
        vetor[i].chave = (rand() % valor_max) + valor_min;
        vetor[i].valor = i;
    }
    free(vetor);
}
\end{lstlisting}
\end{frame}

%%%%%%%%%%%%%%%%%%%%%%%%%%%%%%%%%%%%%%%%%%%%%%%%%%%%%%%%%%%%%%%%%%%%%%%%%%%%%%%%%%%%%%%%%%%%%%%%%%%%%%%%%%%%%

\begin{frame}[fragile]{Exercício 6}{Ordenando um Registro}
\begin{itemize}
\item Agora vamos adaptar o algoritmo de ordenacão {\bf Bubblesort}.
\item O ponto chave é alterar as linhas 4 e 5.
\end{itemize}

\end{frame}

%%%%%%%%%%%%%%%%%%%%%%%%%%%%%%%%%%%%%%%%%%%%%%%%%%%%%%%%%%%%%%%%%%%%%%%%%%%%%%%%%%%%%%%%%%%%%%%%%%%%%%%%%%%%%

\begin{frame}[fragile]{Exercício 2}{Pseudocódigo}
\begin{algorithm}[H]
\caption{Bubblesort} 
\label{Bubblesort}
\Entrada{Vetor $V[0..n-1]$, tamanho $n$}
\Saida{Vetor $V$ ordenado}
\Inicio{
  \Para {($i \leftarrow 1 \textrm{ até } n - 1$)} {
    \Para {($j \leftarrow 0 \textrm{ até } n - i - 1$)} {
    \textcolor{red} {
    	\Se { ($V[j] > V[j+1]$) } {
			Trocar $V[j] \leftrightarrow V[j+1]$
		}    
    }
  }
}
}
\end{algorithm}
\end{frame}

%%%%%%%%%%%%%%%%%%%%%%%%%%%%%%%%%%%%%%%%%%%%%%%%%%%%%%%%%%%%%%%%%%%%%%%%%%%%%%%%%%%%%%%%%%%%%%%%%%%%%%%%%%%%%

\begin{frame}[fragile]{Exercício 6}{Estrutura de Dados}
\begin{lstlisting}[style=CStyle]
void bubblesort(int n, dado *vetor) {
    int i, j;
    dado aux;
    for(i=1; i<n; i++) {
        for(j=0; j<n-i; j++) {
            if(vetor[j].chave > vetor[j+1].chave)  {
                aux.chave = vetor[j+1].chave ;
                aux.valor = vetor[j+1].valor ;
                vetor[j+1].chave = vetor[j].chave;
                vetor[j+1].valor = vetor[j].valor;                
                vetor[j].chave = aux.chave;
                vetor[j].valor = aux.valor;
            }
        }
    }
}
\end{lstlisting}
\end{frame}

%%%%%%%%%%%%%%%%%%%%%%%%%%%%%%%%%%%%%%%%%%%%%%%%%%%%%%%%%%%%%%%%%%%%%%%%%%%%%%%%%%%%%%%%%%%%%%%%%%%%%%%%%%%%%

\begin{frame}[fragile]{Exercício 6}{Ordenando um Registro}
Voce pode criar uma funcão que troque os elementos.
\end{frame}

%%%%%%%%%%%%%%%%%%%%%%%%%%%%%%%%%%%%%%%%%%%%%%%%%%%%%%%%%%%%%%%%%%%%%%%%%%%%%%%%%%%%%%%%%%%%%%%%%%%%%%%%%%%%%

\begin{frame}[fragile]{Exercício 6}{Funcão Tocar}
\begin{lstlisting}[style=CStyle]
void trocar(dado *d1, dado *d2) {
	dado aux;
	aux.chave = d2->chave;
	aux.valor = d2->valor;
	d2->chave = d1->chave;
	d2->valor = d1->valor;                
	d1->chave = aux.chave;
	d1->valor = aux.valor;
}
\end{lstlisting}
\end{frame}
%%%%%%%%%%%%%%%%%%%%%%%%%%%%%%%%%%%%%%%%%%%%%%%%%%%%%%%%%%%%%%%%%%%%%%%%%%%%%%%%%%%%%%%%%%%%%%%%%%%%%%%%%%%%%

\begin{frame}[fragile]{Exercício 6}{Bubblesort}
\begin{lstlisting}[style=CStyle]
void bubblesort(int n, dado *vetor) {
    int i, j;
    for(i=1; i<n; i++) {
        for(j=0; j<n-i; j++) {
            if(vetor[j].chave > vetor[j+1].chave)  {
                trocar(&vetor[j], &vetor[j+1]);
            }
        }
    }
}
\end{lstlisting}
\end{frame}

%%%%%%%%%%%%%%%%%%%%%%%%%%%%%%%%%%%%%%%%%%%%%%%%%%%%%%%%%%%%%%%%%%%%%%%%%%%%%%%%%%%%%%%%%%%%%%%%%%%%%%%%%%%%%

\begin{frame}[fragile]{Exercício 6}{Ordenando um Registro}
Por fim, vamos adicionar as variáveis para contabilizar o tempo gasto.
\end{frame}

%%%%%%%%%%%%%%%%%%%%%%%%%%%%%%%%%%%%%%%%%%%%%%%%%%%%%%%%%%%%%%%%%%%%%%%%%%%%%%%%%%%%%%%%%%%%%%%%%%%%%%%%%%%%%

\begin{frame}[fragile]{Exercício 6}{Bubblesort}
\begin{lstlisting}[style=CStyle]
clock_t tempoInicial, tempoFinal;
dado *vetor;
double tempoGasto;
tempoInicial = clock(); // Obtem o clock inicial
bubblesort(tamanho_vetor, vetor);
tempoFinal = clock(); // Obtem o clock final
// Calcula o tempo gasto
tempoGasto = (tempoFinal-tempoInicial)*1000 / CLOCKS_PER_SEC; 
\end{lstlisting}
\end{frame}

%%%%%%%%%%%%%%%%%%%%%%%%%%%%%%%%%%%%%%%%%%%%%%%%%%%%%%%%%%%%%%%%%%%%%%%%%%%%%%%%%%%%%%%%%%%%%%%%%%%%%%%%%%%%%

\begin{frame}[plain]
  \titlepage
\end{frame}

\end{document}
