\documentclass[aspectratio=169]{beamer}
\usetheme{Boadilla}
%\usetheme{Warsaw}
%\setbeamercovered{transparent}
\beamertemplatetransparentcoveredhigh
\usepackage[portuges]{babel}
\usepackage[utf8]{inputenc}
\usepackage{lmodern}
\usepackage[T1]{fontenc}
\usepackage{hyperref} 
\usepackage{listings}

\lstset{frame=tb,
  language=C,
  aboveskip=3mm,
  belowskip=3mm,
  showstringspaces=false,
  columns=flexible,
  basicstyle={\small\ttfamily},
  numbers=none,
  breaklines=true,
  breakatwhitespace=true,
  tabsize=3
}

\newcommand{\eng}[1]{\textsl{#1}}
\newcommand{\cod}[1]{\texttt{#1}}

\title[Ponteiros e Alocação Dinâmica]{Algoritmos e Estrutura de Dados II}
\subtitle{Tipos Abstratos de Dados}
\author[Frederico Santos de Oliveira]{prof. Frederico Santos de Oliveira}
\institute[UFMT]{Universidade Federal de Mato Grosso\\ Faculdade de Engenharia}
\date{}

\begin{document}

\begin{frame}[plain]
  \titlepage
\end{frame}

%\section*{Roteiro}

%%%%%%%%%%%%%%%%%%%%%%%%%%%%%%%%%%%%%%%%%%%%%%%%%%%%%%%%%%%%%%%%%%%%%%%%%%%%%%%%%%%%%%%%%%%%%%%%%%%%%%%%%%%%%%%

\begin{frame}
  \frametitle{Agenda}
  \tableofcontents
\end{frame}

%%%%%%%%%%%%%%%%%%%%%%%%%%%%%%%%%%%%%%%%%%%%%%%%%%%%%%%%%%%%%%%%%%%%%%%%%%%%%%%%%%%%%%%%%%%%%%%%%%%%%%%%%%%%%%%
\section{Introdução}
%%%%%%%%%%%%%%%%%%%%%%%%%%%%%%%%%%%%%%%%%%%%%%%%%%%%%%%%%%%%%%%%%%%%%%%%%%%%%%%%%%%%%%%%%%%%%%%%%%%%%%%%%%%%%%%

\begin{frame}
\frametitle{Introdução}
\huge{Qual a diferença entre um algoritmo e um programa?}
\end{frame}

%%%%%%%%%%%%%%%%%%%%%%%%%%%%%%%%%%%%%%%%%%%%%%%%%%%%%%%%%%%%%%%%%%%%%%%%%%%%%%%%%%%%%%%%%%%%%%%%%%%%%%%%%%%%%%%

\begin{frame}{Introdução}
\begin{itemize}
\item Algoritmo
\begin{itemize}
\item Sequência de ações executáveis para a solução de um determinado tipo de problema ``apontando'' para a variável que contém o dado desejado.
\item Exemplo: uma receita de bolo.
\item Em geral, algoritmos trabalham sobre estruturas de dados.
\end{itemize}
\item Estruturas de Dados
\begin{itemize}
\item Conjunto de dados que representa uma situação real.
\item Abstração da realidade.
\item Estruturas de dados e algoritmos estão intimamente ligados.
\end{itemize}
\end{itemize}
\end{frame}

%%%%%%%%%%%%%%%%%%%%%%%%%%%%%%%%%%%%%%%%%%%%%%%%%%%%%%%%%%%%%%%%%%%%%%%%%%%%%%%%%%%%%%%%%%%%%%%%%%%%%%%%%%%%%%%

\begin{frame}{Introdução}{Representação}
\begin{itemize}
\item Dados podem estar representados (estruturados) de diferentes maneiras.
\item Normalmente, a escolha da representação é determinada pelas operações que serão utilizadas sobre eles.
\item Exemplo: números inteiros.
\begin{itemize}
\item Representação por palitinhos: $II + IIII = IIIII$. 
\item Boa para pequenos números (operação simples).=
\item Representação decimal: $1278 + 321 = 1599$.
\item Boa para números maiores (operação complexa).
\end{itemize}
\end{itemize}
\end{frame}

%%%%%%%%%%%%%%%%%%%%%%%%%%%%%%%%%%%%%%%%%%%%%%%%%%%%%%%%%%%%%%%%%%%%%%%%%%%%%%%%%%%%%%%%%%%%%%%%%%%%%%%%%%%%%%%

\begin{frame}{Introdução}{Exemplo}
\begin{itemize}
\item Como representar tempo?
\item Depende das operações a serem realizadas:
\begin{itemize}
\item $time\_t$, tipo inteiro, conta o número de segundos desde 1/1/1970.
\item $struct\_timespec$, struct com dois inteiros, um $time\_t$ e um contador de nanosegundos.
\item $struct\_tm$, vários campos para segundo, minuto, dia, mês, ano, horário de verão, etc..
\item $double$, pra armazenar intervalos de tempo.
\end{itemize}
\end{itemize}
\end{frame}

%%%%%%%%%%%%%%%%%%%%%%%%%%%%%%%%%%%%%%%%%%%%%%%%%%%%%%%%%%%%%%%%%%%%%%%%%%%%%%%%%%%%%%%%%%%%%%%%%%%%%%%%%%%%%%%

\begin{frame}{Introdução}{Programa}
\begin{itemize}
\item Um programa é uma formulação concreta de um algoritmo abstrato, baseado em representações de dados específicas.
\item Os programas são feitos em alguma linguagem que pode ser entendida e seguida pelo 
computador.
\item Linguagem de máquina.
\item Linguagem de alto nível (uso de compilador).
\item Utilizaremos a Linguagem C.
\end{itemize}
\end{frame}

%%%%%%%%%%%%%%%%%%%%%%%%%%%%%%%%%%%%%%%%%%%%%%%%%%%%%%%%%%%%%%%%%%%%%%%%%%%%%%%%%%%%%%%%%%%%%%%%%%%%%%%%%%%%%%%

\begin{frame}{Introdução}{Linguagem C}
\begin{itemize}
\item Criada no início da década de 70 para a programação do 
sistema operacional Unix 
\item Uma das linguagens mais utilizadas no mundo, e serviu como base para outras como C++, C\#, Java e etc. 
\item Filosofia: ``O programador sabe o que está fazendo''.
\end{itemize}
\end{frame}

%%%%%%%%%%%%%%%%%%%%%%%%%%%%%%%%%%%%%%%%%%%%%%%%%%%%%%%%%%%%%%%%%%%%%%%%%%%%%%%%%%%%%%%%%%%%%%%%%%%%%%%%%%%%%%%
\section{Tipos Abstratos de Dados}
%%%%%%%%%%%%%%%%%%%%%%%%%%%%%%%%%%%%%%%%%%%%%%%%%%%%%%%%%%%%%%%%%%%%%%%%%%%%%%%%%%%%%%%%%%%%%%%%%%%%%%%%%%%%%%%

\begin{frame}{Introdução}{Tipos Abstratos de Dados}
\begin{itemize}
\item Encapsulam a representação dos dados e as operações que podem ser realizadas sobre eles
\item Usuário do TAD {\it vs.} programador do TAD
\begin{itemize}
\item Usuário só ``enxerga'' a interface, não a implementação.
\item Não importa se a representação é feita com palitos, números decimais, ou em binário desde que a gente consiga somar, subtrair, multiplicar, etc..
\end{itemize}
\item Os usuários de um TAD só têm acesso às operações disponibilizadas sobre os dados.
\end{itemize}
\end{frame}

%%%%%%%%%%%%%%%%%%%%%%%%%%%%%%%%%%%%%%%%%%%%%%%%%%%%%%%%%%%%%%%%%%%%%%%%%%%%%%%%%%%%%%%%%%%%%%%%%%%%%%%%%%%%%%%

\begin{frame}{Introdução}{Tipos Abstratos de Dados}
Isolamento e Reuso:
\begin{itemize}
\item Podemos modificar a implementação do TAD sem modificar o código que usa o TAD.
\item Também odemos modificar o código que usa o TAD sem modificar a implementação do TAD.
\item O TAD pode ser reaproveitado em vários programas ou módulos.
\end{itemize}
\end{frame}

%%%%%%%%%%%%%%%%%%%%%%%%%%%%%%%%%%%%%%%%%%%%%%%%%%%%%%%%%%%%%%%%%%%%%%%%%%%%%%%%%%%%%%%%%%%%%%%%%%%%%%%%%%%%%%%

\begin{frame}{Introdução}{Tipos Abstratos de Dados}
Implementação:
\begin{itemize}
\item Em linguagens orientadas a objeto, como por exemplo C++ e Java, a implementação de TADs é feita por meio de classes. 
\begin{itemize}
\item Será estudado na disciplina POO.
\end{itemize}
\item Em linguagem estruturadas, como C e Pascal, a implementação de TADs é feita por meio da definições de tipos e implementação de funções.
\item Em C, para definição de TADs, utiliza-se as instruções {\it typedef} e {\it struct}.
\end{itemize}
\end{frame}

%%%%%%%%%%%%%%%%%%%%%%%%%%%%%%%%%%%%%%%%%%%%%%%%%%%%%%%%%%%%%%%%%%%%%%%%%%%%%%%%%%%%%%%%%%%%%%%%%%%%%%%%%%%%%%%
\section{Registros (struct)} 
%%%%%%%%%%%%%%%%%%%%%%%%%%%%%%%%%%%%%%%%%%%%%%%%%%%%%%%%%%%%%%%%%%%%%%%%%%%%%%%%%%%%%%%%%%%%%%%%%%%%%%%%%%%%%%%

\begin{frame}{Introdução}{Registros}
\begin{itemize}
\item Vetores e matrizes 
\begin{itemize}
\item Estruturas de dados homogêneas 
\item Armazenam vários valores, mas todos de um mesmo tipo (todos int, todos double, todos float, todos char).
\end{itemize}
\item Registro (ou {\it struct})
\begin{itemize}
\item Tipo de dado estruturado heterogêneo.
\item Coleção de variáveis referenciadas sobre um mesmo nome.
\item Permite agrupar dados de diferentes tipos numa mesma estrutura (ao contrário de matrizes que possuem elementos de um mesmo tipo).
\item Cada componente de um registro pode ser de um tipo diferente (int, char, ...)
\end{itemize}
\end{itemize}
\end{frame}

%%%%%%%%%%%%%%%%%%%%%%%%%%%%%%%%%%%%%%%%%%%%%%%%%%%%%%%%%%%%%%%%%%%%%%%%%%%%%%%%%%%%%%%%%%%%%%%%%%%%%%%%%%%%%%%

\begin{frame}{Introdução}{Registros}
Registros ({\it struct}):
\begin{itemize}
\item Uma estrutura é uma coleção de uma ou mais variáveis colocadas juntas sob um único nome para manipulação conveniente. 
\item Por exemplo, para representar um aluno são necessárias as informações nome, matrícula, conceito. 
\item Ao invés de criar três variáveis, é possível criar uma única variável contendo três campos. 
\item Em C, usa-se a instrução {\it struct} para representar esse tipo de dado.
\end{itemize}
\end{frame}

%%%%%%%%%%%%%%%%%%%%%%%%%%%%%%%%%%%%%%%%%%%%%%%%%%%%%%%%%%%%%%%%%%%%%%%%%%%%%%%%%%%%%%%%%%%%%%%%%%%%%%%%%%%%%%%

\begin{frame}{Introdução}{Registros}
Conceito de {\it struct}:
\begin{itemize}
\item Os elementos do registro são chamados de campos ou membros da struct.
\item É utilizado para armazenar informações de um mesmo objeto.
\item Exemplos: 
\begin{itemize}
\item carro $\rightarrow$ cor, marca, ano, placa, chassi.
\item pessoa $\rightarrow$ nome, idade, endereço.
\end{itemize}
\end{itemize}
\end{frame}

%%%%%%%%%%%%%%%%%%%%%%%%%%%%%%%%%%%%%%%%%%%%%%%%%%%%%%%%%%%%%%%%%%%%%%%%%%%%%%%%%%%%%%%%%%%%%%%%%%%%%%%%%%%%%%%
\section{Exemplos}
%%%%%%%%%%%%%%%%%%%%%%%%%%%%%%%%%%%%%%%%%%%%%%%%%%%%%%%%%%%%%%%%%%%%%%%%%%%%%%%%%%%%%%%%%%%%%%%%%%%%%%%%%%%%%%%

\begin{frame}[fragile]{Registros}{Exemplo}
Exemplo de um registro para armazenar dados de um ponto em duas dimensões:
\begin{lstlisting}
struct ponto {
  float x ;
  float y ;
};
struct ponto p1;
p1.x = 0.0;
p1.y = 0.0;
\end{lstlisting} 
\end{frame}

%%%%%%%%%%%%%%%%%%%%%%%%%%%%%%%%%%%%%%%%%%%%%%%%%%%%%%%%%%%%%%%%%%%%%%%%%%%%%%%%%%%%%%%%%%%%%%%%%%%%%%%%%%%%%%%

\begin{frame}[fragile]{Registros}{Exemplo}
Exemplo de um registro para armazenar uma fração:
\begin{lstlisting}
struct fracao {
  int numerador;
  int denominador;
};
struct fracao f1;
f1.numerador   = 1;
f1.denominador = 2;
\end{lstlisting} 
\end{frame}

%%%%%%%%%%%%%%%%%%%%%%%%%%%%%%%%%%%%%%%%%%%%%%%%%%%%%%%%%%%%%%%%%%%%%%%%%%%%%%%%%%%%%%%%%%%%%%%%%%%%%%%%%%%%%%%

\begin{frame}[fragile]{Registros}{Exemplo}
Exemplo de um registro para armazenar dados de um aluno:
\begin{lstlisting}
struct aluno {
  char nome[30];
  int matricula;
};
struct aluno aluno_jose;
strcpy(aluno_jose.nome,"Jose dos Santos");
aluno_jose.matricula = 20169999; 
\end{lstlisting} 
\end{frame}

%%%%%%%%%%%%%%%%%%%%%%%%%%%%%%%%%%%%%%%%%%%%%%%%%%%%%%%%%%%%%%%%%%%%%%%%%%%%%%%%%%%%%%%%%%%%%%%%%%%%%%%%%%%%%%%

\begin{frame}[fragile]{Registros}{Exemplo}
Exemplo de um registro para armazenar uma data:
\begin{lstlisting}
struct data {
  int dia;
  int mes;
  int ano;
};
struct data natal;
natal.dia = 25;
natal.mes = 12;
natal.ano = 2017;
\end{lstlisting} 
\end{frame}

%%%%%%%%%%%%%%%%%%%%%%%%%%%%%%%%%%%%%%%%%%%%%%%%%%%%%%%%%%%%%%%%%%%%%%%%%%%%%%%%%%%%%%%%%%%%%%%%%%%%%%%%%%%%%%%

\begin{frame}[fragile]{Registros}{Exemplo}
Exemplo de um registro para armazenar dados de um aluno, que é formado por um campo do tipo registro:
\begin{lstlisting}
struct data {
  int dia;
  int mes;
  int ano;
};
struct aluno {
  char nome[30];
  int matricula;
  struct data nasc;
};
struct aluno aluno1;
strcpy(aluno1.nome,"Jose dos Santos");
aluno1.matricula = 20169999; 
aluno1.nasc.dia  = 1;
aluno1.nasc.mes = 1;
aluno1.nasc.ano = 2000;
\end{lstlisting} 
\end{frame}

%%%%%%%%%%%%%%%%%%%%%%%%%%%%%%%%%%%%%%%%%%%%%%%%%%%%%%%%%%%%%%%%%%%%%%%%%%%%%%%%%%%%%%%%%%%%%%%%%%%%%%%%%%%%%%%
\section{Declaração de Tipos (typedef)}
%%%%%%%%%%%%%%%%%%%%%%%%%%%%%%%%%%%%%%%%%%%%%%%%%%%%%%%%%%%%%%%%%%%%%%%%%%%%%%%%%%%%%%%%%%%%%%%%%%%%%%%%%%%%%%%

\begin{frame}{Declaração de Tipos}{Declaração de Tipos}
\begin{itemize}
\item Para simplificar, uma estrutura ou mesmo outros tipos de dados podem ser definidos como um novo tipo.
\item Uso da construção {\it typedef}
\end{itemize}
\end{frame}	

%%%%%%%%%%%%%%%%%%%%%%%%%%%%%%%%%%%%%%%%%%%%%%%%%%%%%%%%%%%%%%%%%%%%%%%%%%%%%%%%%%%%%%%%%%%%%%%%%%%%%%%%%%%%%%%
\section{Exemplos}
%%%%%%%%%%%%%%%%%%%%%%%%%%%%%%%%%%%%%%%%%%%%%%%%%%%%%%%%%%%%%%%%%%%%%%%%%%%%%%%%%%%%%%%%%%%%%%%%%%%%%%%%%%%%%%%

\begin{frame}[fragile]{Declaração de Tipos}{Exemplo}
Exemplo da definição do tipo livro:
\begin{lstlisting}
typedef struct {
   char titulo[50];
   char autor[50];
   char assunto[100];
   int codigo;
} livro;
  \end{lstlisting} 
\end{frame}

%%%%%%%%%%%%%%%%%%%%%%%%%%%%%%%%%%%%%%%%%%%%%%%%%%%%%%%%%%%%%%%%%%%%%%%%%%%%%%%%%%%%%%%%%%%%%%%%%%%%%%%%%%%%%%%

\begin{frame}[fragile]{Declaração de Tipos}{Exemplo}
Exemplo de uso do tipo livro:
\begin{lstlisting}
   livro book1;   
   strcpy( book1.titulo, "C Como Programar");
   strcpy( book1.autor, "H. M. Deitel"); 
   strcpy( book1.assunto, "Programacao em c");
   book1.codigo = 123456;
   printf( "Titulo do livro : %s\n", book1.titulo);
   printf( "Autor do livro : %s\n", book1.autor);
   printf( "Assunto do livro : %s\n", book1.assunto);
   printf( "Codigo do livro: %d\n", book1.codigo);
  \end{lstlisting} 
\end{frame}

%%%%%%%%%%%%%%%%%%%%%%%%%%%%%%%%%%%%%%%%%%%%%%%%%%%%%%%%%%%%%%%%%%%%%%%%%%%%%%%%%%%%%%%%%%%%%%%%%%%%%%%%%%%%%%%

\begin{frame}[fragile]{Declaração de Tipos}{Exemplo}
Exemplo de uso de ponteiro com registro:
\begin{lstlisting}
   livro *book2;
   book2 = malloc(sizeof(livro));
   strcpy( book2->titulo, "Projeto de Algoritmos");
   strcpy( book2->autor, "Nivio Ziviani"); 
   strcpy( book2->assunto, "Estruturas de dados");
   book2->codigo = 999999;
   printf( "Titulo do livro : %s\n", book2->titulo);
   printf( "Autor do livro : %s\n", book2->autor);
   printf( "Assunto do livro : %s\n", book2->assunto);
   printf( "Codigo do livro: %d\n", book2->codigo);
\end{lstlisting} 
\end{frame}

%%%%%%%%%%%%%%%%%%%%%%%%%%%%%%%%%%%%%%%%%%%%%%%%%%%%%%%%%%%%%%%%%%%%%%%%%%%%%%%%%%%%%%%%%%%%%%%%%%%%%%%%%%%%%%%
\section{Tipos Abstratos de Dados}
%%%%%%%%%%%%%%%%%%%%%%%%%%%%%%%%%%%%%%%%%%%%%%%%%%%%%%%%%%%%%%%%%%%%%%%%%%%%%%%%%%%%%%%%%%%%%%%%%%%%%%%%%%%%%%%

\begin{frame}{Tipos Abstratos de Dados}{Declaração de Tipos}
\begin{itemize}
\item Para implementar um Tipo Abstrato de Dados em C, usa-se a definição de tipos juntamente com a implementação de funções que agem sobre aquele tipo. 
\item Como boa regra de programação, evita-se acessar o dado diretamente, fazendo o acesso 
somente através das funções. 
\end{itemize}
\end{frame}	

%%%%%%%%%%%%%%%%%%%%%%%%%%%%%%%%%%%%%%%%%%%%%%%%%%%%%%%%%%%%%%%%%%%%%%%%%%%%%%%%%%%%%%%%%%%%%%%%%%%%%%%%%%%%%%%
\section{Exemplos}
%%%%%%%%%%%%%%%%%%%%%%%%%%%%%%%%%%%%%%%%%%%%%%%%%%%%%%%%%%%%%%%%%%%%%%%%%%%%%%%%%%%%%%%%%%%%%%%%%%%%%%%%%%%%%%%

\begin{frame}[fragile]{Tipos Abstratos de Dados}{Exemplo}
TAD para representar números complexos:
\begin{lstlisting}
typedef struct {
  double real;
  double img;
} complexo;
\end{lstlisting} 
\end{frame}

%%%%%%%%%%%%%%%%%%%%%%%%%%%%%%%%%%%%%%%%%%%%%%%%%%%%%%%%%%%%%%%%%%%%%%%%%%%%%%%%%%%%%%%%%%%%%%%%%%%%%%%%%%%%%%%

\begin{frame}[fragile]{Tipos Abstratos de Dados}{Exemplo}
Função que soma dois  números complexos:
\begin{lstlisting}
complexo *soma(complexo *n1, complexo *n2) {
  complexo *soma = malloc(sizeof(complexo));
  soma->real = n1->real + n2->real;
  soma->img = n1->img + n2->img;
  return soma;
}
\end{lstlisting} 
\end{frame}

%%%%%%%%%%%%%%%%%%%%%%%%%%%%%%%%%%%%%%%%%%%%%%%%%%%%%%%%%%%%%%%%%%%%%%%%%%%%%%%%%%%%%%%%%%%%%%%%%%%%%%%%%%%%%%%

\begin{frame}[fragile]{Tipos Abstratos de Dados}{Exemplo}
Função que multiplica dois números complexos:
\begin{lstlisting}
complexo *multiplica(complexo *n1, complexo *n2) {
  complexo *mul = malloc(sizeof(complexo));
  mul->real = n1->real * n2->real - n1->img * n2->img;
  mul->img = n1->real * n2->img + n1->img * n2->real;
  return mul;
}
\end{lstlisting} 
\end{frame}

%%%%%%%%%%%%%%%%%%%%%%%%%%%%%%%%%%%%%%%%%%%%%%%%%%%%%%%%%%%%%%%%%%%%%%%%%%%%%%%%%%%%%%%%%%%%%%%%%%%%%%%%%%%%%%%

\begin{frame}[fragile]{Tipos Abstratos de Dados}{Exemplo}
Função que inverte um número complexo:
\begin{lstlisting}
complexo *inverte(complexo *n) {
  complexo *inv = malloc(sizeof(complexo));
  inv->real = (-1) * n->real;
  inv->img = (-1) * n->img;
  return inv;
}
\end{lstlisting} 
\end{frame}

%%%%%%%%%%%%%%%%%%%%%%%%%%%%%%%%%%%%%%%%%%%%%%%%%%%%%%%%%%%%%%%%%%%%%%%%%%%%%%%%%%%%%%%%%%%%%%%%%%%%%%%%%%%%%%%

\begin{frame}[fragile]{Tipos Abstratos de Dados}{Exemplo}
Funções que inicializa e imprime números complexos:
\begin{lstlisting}
\begin{lstlisting}
void inicializa(complexo *n, double r, double i) {
  n ->real = r;
  n ->img = i;
}

void escreve(complexo *n) {
  printf("%.2f + %.2fi\n", n->real, n->img);
}
\end{lstlisting} 
\end{frame}

%%%%%%%%%%%%%%%%%%%%%%%%%%%%%%%%%%%%%%%%%%%%%%%%%%%%%%%%%%%%%%%%%%%%%%%%%%%%%%%%%%%%%%%%%%%%%%%%%%%%%%%%%%%%%%%

\begin{frame}{Tipos Abstratos de Dados}{Exercício}
Implemente um TAD conta bancária com campos número e saldo que suporte as as seguintes operações:
\begin{itemize}
\item Iniciar uma nova conta com um número e saldo.
\item Depositar um valor na conta.
\item Sacar um valor da conta.
\item Imprimir o saldo.
\end{itemize}
\end{frame}	

%%%%%%%%%%%%%%%%%%%%%%%%%%%%%%%%%%%%%%%%%%%%%%%%%%%%%%%%%%%%%%%%%%%%%%%%%%%%%%%%%%%%%%%%%%%%%%%%%%%%%%%%%%%%%%%

\begin{frame}
  \frametitle{FIM}
\huge{Fim}
\end{frame}	


\begin{frame}[plain]
  \titlepage
\end{frame}
\end{document}
